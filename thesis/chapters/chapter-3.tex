% -*- TeX-master: "../main.tex" -*-

\chapter{Method}%
\label{chap:method}

This is an introduction to the chapter.

\section{The Algorithm}%
\label{sec:the-algorithm}

The height bound of Theorem~\ref{thm:height-bound} is the fundamental result that allows the \(S\)-unit equation to be solved algorithmically. For an \(S\)-unit \(f\), we have by definition that \(v_{P}(f) = 0\) for all \(P \notin S\). This means that the height of \(f\) is determined by only the finite set of valuations on \(S\). These valuations are discrete, taking on values in \(\mathbb{Z}\). As a result, there are only a finite number of combinations of values for the valuations on \(S\) satisfying a given height bound. Up to a constant factor, each such combination of values corresponds to a unique \(S\)-unit. In this manner, Theorem~\ref{thm:height-bound} limits the space of solutions to the \(S\)-unit equation to a finite set, thus making it approachable by an exhaustive algorithmic search.

In order to determine all \(S\)-units satisfying a given height bound we use the theory of divisors and Riemann-Roch spaces discussed in Section~\ref{sec:divisors-and-riemann-roch-spaces}. Let \(S \subset \mathbb{P}_{K}\) be a finite set of places on the function field \(K\), and let
\[N = 2g_{K} - 2 + \sum_{P \in S} \deg{P}\]
be the bound of Theorem~\ref{thm:height-bound}. Given a positive integer \(n \leq N\), we are interested in finding an \(S\)-unit \(f\) such that \(H(f) = n\). Note that the divisor of \(f\) is a formal sum of places in \(S\). More precisely,
\[(f) = \sum_{P \in S} v_{P}(f) P.\]
If \(S_{0} \subset S\) denotes the zeros of \(f\) and \(S_{\infty} \subset S\) denotes the poles, then we define the \textit{zero divisor} \({(f)}_{0}\) and the \textit{pole divisor} \({(f)}_{\infty}\) by
\[{(f)}_{0} := \sum_{P \in S_{0}} v_{P}(f) P \qquad \text{and} \qquad {(f)}_{\infty} := \sum_{P \in S_{\infty}} - v_{P}(f) P.\]
The requirement that \(H(f) = n\) is then equivalent to saying that \(\deg{{(f)}_{0}} = \deg{{(f)}_{\infty}} = n\).

Reversing this line of reasoning, we are interested in finding integers \(n_{P}\) for \(P \in S\) giving rise to a divisor
\[D = \sum_{P \in S} n_{P} P\]
such that \(\deg{D_{0}} = \deg{D_{\infty}} = n\). Here
\[D_{0} := \sum_{P \in S_{0}} n_{P} P \qquad \text{and} \qquad D_{\infty} := \sum_{P \in S_{\infty}} - n_{P} P,\]
with \(S_{0} \subset S\) denoting the places where \(n_{P} > 0\) and \(S_{\infty} \subset S\) denoting those where \(n_{P} < 0\). Given such \(n_{P}\), we can construct the Riemann-Roch space \(\mathcal{L}(D)\). Following the discussion of Theorem~\ref{thm:riemann-roch-space-properties}, if \(\dim{\mathcal{L}(D)} = 1\) with the single basis vector \(g \in K\), then we can take \(f = g^{-1}\), and by construction \(f\) is an \(S\)-unit satisfying \(H(f) = n\).

Generating the \(S\)-units \(f\) with \(H(f) \leq N\) can thus be reduced to finding all combinations of integers \(n_{P}\) for \(P \in S\) such that
\[\sum_{P \in S_{0}} n_{P} \cdot \deg{P} = \sum_{P \in S_{\infty}} - n_{P} \cdot \deg{P} = n\]
for all positive integers \(n \leq N\). This is similar to the problem of finding \textit{integer partitions} of \(n\). An integer partition of \(n\) into \(m\) parts is a decreasing sequence of positive integers \((x_{1}, x_{2}, \dots, x_{m})\) such that
\[x_{1} + x_{2} + \cdots + x_{m} = n.\]
In our case, we need two integer partitions \((x_{1}, x_{2}, \dots, x_{m})\) and \((y_{1}, y_{2}, \dots, y_{m'})\) of \(n\) with \({m + m' \leq |S|}\). Then, for any order on the places \(S\) such that we can write \(S = \{P_{1}, P_{2}, \dots, P_{|S|}\}\) and \(n_{i} := n_{P_{i}}\), we choose the \(n_{i}\) such that \(n_{i} \cdot \deg{P_{i}} = x_{i}\) for \(i \leq m\), \(n_{i} \cdot \deg{P_{i}} = -y_{i - m}\) for \(m < i \leq m'\), and \(n_{i} = 0\) otherwise. This is possible provided \(\deg{P_{i}} \mid x_{i}\) and \(\deg{P_{j}} \mid y_{j - m}\).

The process of determining the integers \(n_{i}\) is described in Algorithm~\ref{alg:bounded-values}. It considers all appropriately sized pairs of partitions of the bound \(n\). Each pair of partitions then gets merged into a sequence \(v\), which is padded with zeros so that it is of length \(|S|\). The algorithm then evaluates every unique permutation of the integers in \(v\), dividing each integer in the permutation by the degree of the corresponding place. The resulting sequence of values is recorded provided each division resulted in an integer.

\begin{algorithm}[htb]
  \caption{An algorithm for generating all possible values for the valuations of an \(S\)-unit of height \(n\). The provided set of places is assumed to be ordered, and we write \(S = \{P_{1}, P_{2}, \dots, P_{|S|}\}\). The result is a set of sequences of values, with the order of the sequences matching those of the places.}%
  \label{alg:bounded-values}
  \begin{algorithmic}[1]
    \Function{Bounded-Values}{$S, n$}
    \State{$V \gets \{\}$} \Comment{Sequences of values found so far.}
    \For{$m = 1, \dots, |S| - 1$}
    \ForAll{$(x_{1}, \dots, x_{m}) \in$ \Call{Partitions}{$n, m$}}
    \For{$m' = 1, \dots, |S| - m$}
    \ForAll{$(y_{1}, \dots, y_{m'}) \in$ \Call{Partitions}{$n, m'$}}
    \State{$v \gets (\underbrace{x_{1}, \dots, x_{m}, -y_{1}, \dots, -y_{m}, 0, \dots, 0}_{|S|\ \text{elements}})$}
    \ForAll{$(n_{1}, \dots, n_{|S|}) \in \Call{Permutations}{v}$}
    \For{$i = 1, \dots, |S|$}
    \State{$n_{i} \gets n_{i} / \deg{P_{i}}$}
    \EndFor{}
    \If{$n_{i} \in \mathbb{Z}\ \text{for all}\ i \in \{1, \dots, |S|\}$}
    \State{$V \gets V \cup (n_{1}, \dots, n_{|S|})$}
    \EndIf{}
    \EndFor{}
    \EndFor{}
    \EndFor{}
    \EndFor{}
    \EndFor{}
    \State{\Return{$V$}}
    \EndFunction{}
  \end{algorithmic}
\end{algorithm}

We can run Algorithm~\ref{alg:bounded-values} for the positive integers \(n \leq N\) to generate every sequence of values corresponding to the valuations of a bounded \(S\)-unit. Each sequence of values leads to a divisor and thus to a Riemann-Roch space. The \(S\)-unit can then be read from the basis of the Riemann-Roch space, provided the space is of dimension 1. This process, which generates every bounded \(S\)-unit, is described in Algorithm~\ref{alg:bounded-s-units}.

\begin{algorithm}[htb]
  \caption{An algorithm for generating all \(S\)-units whose height is bounded by \(N\).}%
  \label{alg:bounded-s-units}
  \begin{algorithmic}[1]
    \Function{Bounded-S-Units}{$S, N$}
    \State{$U \gets \{\}$} \Comment{$S$-units found so far.}
    \For{$n = 1, \dots, N$}
    \ForAll{$(n_{1}, \dots, n_{|S|}) \in$ \Call{Bounded-Values}{$S, n$}}
    \State{$D \gets \sum_{i = 1}^{|S|} n_{i} P_{i}$} \Comment{Construct divisor.}
    \If{$\dim{\mathcal{L}(D) = 1}$}
    \State{$\{f\} \gets \Call{Basis}{\mathcal{L}(D)}$}
    \State{$U \gets U \cup f^{-1}$} \Comment{Found a new $S$-unit.}
    \EndIf{}
    \EndFor{}
    \EndFor{}
    \State{\Return{$U$}}
    \EndFunction{}
  \end{algorithmic}
\end{algorithm}

With the ability to generate all \(S\)-units satisfying the height bound of Theorem~\ref{thm:height-bound}, it is relatively straightforward to find all solutions to the \(S\)-unit equation. Consider any pair of \(S\)-units \(f, g \in K\). We are interested in finding constants \(a, b \in k\) such that
\[af + bg = 1.\]
We can differentiate this equation to obtain
\[af' + bg' = 0.\]
Since \(f\) and \(g\) are \(S\)-units, their derivatives \(f'\) and \(g'\) are nonzero and these two equations form a linear system in the constants \(a\) and \(b\) with at most one solution. Provided a solutions exists, it immediately leads to a solution to the \(S\)-unit equation.

\section{Complexity and Performance}%
\label{sec:complexity-and-performance}

The algorithm presented in Section~\ref{sec:the-algorithm} is effectively a brute force search through the space of bounded \(S\)-units. This is simple to implement, but it is not very efficient since the number of bounded \(S\)-units can quickly grow to be quite large. To see why this is the case, we construct an upper bound on the number of \(S\)-units produced by Algorithm~\ref{alg:bounded-s-units}, depending on the size of the set \(S\) and the height bound \(N\).

First, we note that the number of partitions of the integer \(n\) is given by the \textit{partition function} \(p(n)\). The \textit{restricted partition function} \(p_{m}(n)\) yields the number of partitions of \(n\) into \(m\) parts. There is no known closed-form expression for either function. However, it can be shown that (see e.g.~\cite[Theorem 6.3]{andrews-1984-the-theory-of})
\[p(n) \sim \frac{e^{\pi\sqrt{\frac{2n}{3}}}}{4n\sqrt{3}} \quad \text{as} \quad n \to \infty.\]

Algorithm~\ref{alg:bounded-values} considers all pairs of partitions  of \(n\) into \(m\) and \(m'\) parts with \(m + m' \leq |S|\). The number of such pairs is given by
\[\sum_{m = 1}^{|S| - 1} \left( p_{m}(n) \sum_{m' = 1}^{|S| - m} p_{m'}(n) \right).\]
This expression is somewhat cumbersome. We can simplify our analysis by ignoring the fact that the partitions in Algorithm~\ref{alg:bounded-values} are restricted. There are of course fewer pairs of restricted partitions than pairs of unrestricted partitions. Therefore, \({p(n)}^{2}\) provides a simple upper bound on the number of pairs considered by the algorithm.

Algorithm~\ref{alg:bounded-values} merges each pair of partitions into a sequence of \(|S|\) integers and considers all unique permutations of this sequence. There are at most \(|S|!{}\) unique permutations, so \({p(n)}^{2}|S|!{}\) is an upper bound on the number of sequences produced by Algorithm~\ref{alg:bounded-values}.

In order to generate all \(S\)-units whose height is bounded by \(N\), Algorithm~\ref{alg:bounded-s-units} considers the sequences produced by Algorithm~\ref{alg:bounded-values} for \(n = 1, 2, \dots, N\). Because of the rapid growth of the partition function, the number of sequences is dominated by those for the highest bound \(n = N\). As a result, the number of bounded \(S\)-units generated by Algorithm~\ref{alg:bounded-s-units} is in the order of \({p(N)}^{2}|S|!{}\).

Solving the \(S\)-unit equation requires considering every pair of bounded \(S\)-units. The total number of candidate solutions is therefore in the order of \({({p(N)}^{2}|S|!{})}^{2}\). This grows quickly in both \(|S|\) and \(N\), leaving \(S\)-unit equations for large sets \(S\) and high height bounds \(N\) practically impossible to solve. We note, however, that we do not need to consider every pair. The order of the \(S\)-units in the \(S\)-unit equation does not matter. Additionally, if the pair \((f, g)\) solves the \(S\)-unit equation, then so do \((f^{-1}, - f^{-1}g)\) and \((g^{-1}, - g^{-1}f)\). In practice, we therefore only need to consider one out of every six pairs of \(S\)-units. This does not change the asymptotic behaviour of the algorithm, but it does provide a meaningful improvement in performance.

We can further limit the number of pairs of \(S\)-units the algorithm needs to consider by using the fact that in order to solve the \(S\)-unit equation two \(S\)-units must be of the same height. To see why this is the case, note that from property~\ref{prop:valuation-addition} of Theorem~\ref{thm:valuation-properties} it follows that for all \(f, g \in K\) we have
\[H(f + g) \leq H(f) + H(g).\]
Therefore, if \(f + g = 1\), then
\[H(f) = H(1 - g) \leq H(1) + H(g) = H(g)\]
and similarly
\[H(g) = H(1 - f) \leq H(1) + H(f) = H(f).\]
We can conclude that \(H(f) = H(g)\). As with the previous limits on the number of pairs \(S\)-units, this does not alter the time complexity of the algorithm. However, in practice it noticeably speeds up the search for solutions.
