% -*- TeX-master: "../main.tex" -*-

\chapter{Background}%
\label{chap:background}

Throughout this chapter we denote by \(k\) an arbitrary field. We write \(k(x)\) for the rational function field over \(k\), and we consider a finite algebraic extension \(K\) of \(k(x)\).\todo{Add an overview of upcoming sections.}\todo{Perhaps mention that the first two section closely follow Chapter 1 in~\cite{stichtenoth-2009-algebraic-function-fields} and/or refer the reader to that book for more details.}

\section{Places and Valuations}%
\label{sec:places-and-valuations}

The related notions of places and valuations of function fields are fundamental to the formulation of the \(S\)-unit equation. We first define the concept of a valuation ring.

\begin{definition}%
  \label{def:valuation-ring}
  A valuation ring of the function field \(K / k(x)\) is a ring \(\mathcal{O} \subseteq K\) with \(k \subset \mathcal{O} \subset K\) and for every \(f \in K\) at least one of \(f\) or \(f^{-1}\) is in \(\mathcal{O}\).
\end{definition}

In the rational function field \(k(x)\) we can construct valuation rings by considering an irreducible sonic polynomial \(p \in k[x]\) and taking
\[\mathcal{O}_{p} = \left\{ \frac{f}{g} \,\middle|\, f,g \in k[x] ,\, p \nmid g \right\}.\]
To see that this satisfies Definition~\ref{def:valuation-ring} we note that every element of \(k(x)\) can be written as a reduced fraction \(f/g\) with \(f,g \in k[x]\) and \(p \nmid f\) or \(p \nmid g\). Therefore we always have \(f/g \in \mathcal{O}_{p}\) or \(g/f \in \mathcal{O}_{p}\). We note that a different irreducible monic polynomial \(q \in k[x]\) gives rise to a different valuation ring \(\mathcal{O}_{q}\). For example, we have \(1/q \in \mathcal{O}_{p}\) but \(1/q \notin \mathcal{O}_{q}\).

The units of a valuation ring \(\mathcal{O}\) of the function field \(K\) are those \(f \in K\) such that \(f \in \mathcal{O}\) as well as \(f^{-1} \in \mathcal{O}\). For \(\mathcal{O}_{p}\) in the rational function field \(k(x)\) defined by an irreducible monic polynomial \(p \in k[x]\) this corresponds to the ratios of polynomials which do not contain a factor \(p\) in the numerator nor in the denominator. More precisely,
\[\mathcal{O}_{p}^{\times} = \left\{ \frac{f}{g} \,\middle|\, f,g \in k[x] ,\, p \nmid f ,\, p \nmid g \right\}.\]
This means that the ideal generated by \(p\) is given by \((p) = \mathcal{O}_{p} \setminus \mathcal{O}_{p}^{\times}\). It follows that \((p)\) is the unique maximal ideal of \(\mathcal{O}_{p}\) since any proper ideal \(I \subset \mathcal{O}_{p}\) cannot contain a unit.

This property is not unique to valuation rings of the rational function field generated by an irreducible monic polynomial. The following theorem is a reformulation of Proposition 1.1.5 in~\cite{stichtenoth-2009-algebraic-function-fields}.

\begin{theorem}%
  \label{thm:place-is-maximal-ideal}
  A valuation ring \(\mathcal{O}\) of the function field \(K\) has a unique maximal ideal \(P = \mathcal{O} \setminus \mathcal{O}^{\times}\).
\end{theorem}

The unique maximal ideal \(P\) of a valuation ring \(\mathcal{O}\) is called a \textit{place}. The set of all places in the function field \(K\) is denoted by \(\mathbb{P}_{K}\). Given a place \(P\), we write \(\mathcal{O}_{P}\) for its corresponding valuation ring.

Since each place is a maximal ideal, the factor ring \(\mathcal{O}_{P} / P\) is a field. By Definition~\ref{def:valuation-ring} we have \(k \subset \mathcal{O}_{P}\), and since \(P\) does not contain units we also have \(k \cap P = \{0\}\). It follows that \(\mathcal{O}_{P} / P\) extends the field of constants \(k\). This allows us to define the \textit{degree} of a place as the degree of this extension. More precisely, we write \(\deg{P} := [\mathcal{O}_{P} / P : k]\).

We saw that in the rational function field \(k(x)\) the irreducible monic polynomials correspond to distinct places and thus to distinct valuation rings. These are not the only places in \(k(x)\), however. There is one more valuation ring given by
\[\mathcal{O}_{\infty} = \left\{ \frac{f}{g} \,\middle|\, f,g \in k[x] ,\, \deg{f} \leq \deg{g} \right\},\]
which has the maximal ideal
\[P_{\infty} = \left\{ \frac{f}{g} \,\middle|\, f,g \in k[x] ,\, \deg{f} < \deg{g} \right\}.\]
The place \(P_{\infty}\) is referred to as the \textit{infinite place}. Note that \(P_{\infty} = (1/x)\), so the infinite place is a principal ideal much like the other places.

This, again, is a property common to valuation rings and places of all function fields, not just those of the rational function field. The following theorem is a reformulation of Theorem 1.1.6 in~\cite{stichtenoth-2009-algebraic-function-fields}.

\begin{theorem}%
  \label{thm:place-is-principal-ideal}
  In a function field \(K\) every place \(P \in \mathbb{P}_{K}\) is a principal ideal. Moreover, for a \(p \in \mathcal{O}_{P}\) such that \((p) = P\) we have that any \(f \in K\) can be uniquely written as \(f = p^{n}u\) with \(n \in \mathbb{Z}\) and \(u \in \mathcal{O}_{P}^{\times}\).
\end{theorem}

An element \(p\) which generates the place \(P\) is called a \textit{uniformizer} of \(P\). In the rational function field \(k(x)\) a irreducible monic polynomial is a uniformizer of its corresponding place.

We note that the integer \(n\) in the second part of Theorem~\ref{thm:place-is-principal-ideal} is unique for a given place \(P \in \mathbb{P}_{K}\) and element \(f \in K\). If \(p\) and \(q\) are distinct uniformizers of \(P\) then we have \(p = qs\) and \(q = pt\) for \(s,t \in \mathcal{O}_{P}\). In fact, \(s,t \in \mathcal{O}_{P}^{\times}\) since \(s = p/q\) and \(t = q/p\) are each other's inverses. It follows that if \(f = p^{n}u\) with \(u \in \mathcal{O}_{P}^{\times}\) then also \(f = q^{n}s^{n}u\) with \(s^{n}u \in \mathcal{O}_{P}^{\times}\). This justifies the following definition.

\begin{definition}%
  \label{def:valuation}
  For a place \(P \in \mathbb{P}_{K}\) with a uniformizer \(p\) we construct the function \(v_{P} : K \to \mathbb{Z} \cup \{\infty\}\) by considering for each non-zero \(f \in K\) the unique decomposition \(f = p^{n}u\) as in Theorem~\ref{thm:place-is-principal-ideal} and defining \(v_{P}(f) := n\). Additionally, we define \({v_{P}(0) := \infty}\).
\end{definition}

The function \(v_{P}\) is the \textit{valuation} associated with the place \(P\). As mentioned earlier, in the rational function field \(k(x)\) a finite place \(P\) has a uniformizer \(p\) which is an irreducible monic polynomial. For a rational function \(f \in k(x)\) the valuation \(v_{P}\) then counts the number of times \(p\) divides the numerator and denominator of \(f\), with occurrences in the numerator contributing positively and occurrences in the denominator contributing negatively.

Valuations as defined above have many nice properties. The following theorem is a reformulation of Theorem 1.1.13 in~\cite{stichtenoth-2009-algebraic-function-fields}.

\begin{theorem}%
  \label{thm:valuation-properties}
  In a function field \(K / k(x)\) the valuation \(v_{P}\) for a place \(P \in \mathbb{P}_{K}\) as given by Definition~\ref{def:valuation} has the following properties:
  \begin{enumerate}[label = {(\arabic*)}]
    \item%
      \label{prop:valuation-multiplication}
      \(v_{P}(fg) = v_{P}(f) + v_{P}(g)\).

    \item%
      \label{prop:valuation-addition}
      \(v_{P}(f + g) \geq \min{\left\{v_{P}(f), v_{P}(g)\right\}}\).

    \item%
      \label{prop:valuation-zero}
      \(v_{P}(f) = \infty\) if and only if \(f = 0\).

    \item
      \(v_{P}(a) = 0\) for all \(a \in k\).

    \item
      \(v_{P}(f) = 1\) if and only if \(f\) is a uniformizer of \(P\).

    \item
      We have
      \begin{align*}
        \mathcal{O}_{P} = \left\{ f \in K \,\middle|\, v_{P}(f) \geq 0 \right\},       \\
        \mathcal{O}_{P}^{\times} = \left\{ f \in K \,\middle|\, v_{P}(f) = 0 \right\}, \\
        P = \left\{ f \in K \,\middle|\, v_{P}(f) > 0 \right\}.
      \end{align*}
  \end{enumerate}
\end{theorem}

We note that Definition~\ref{def:valuation} is not the only approach to defining valuations on function fields. It is also possible to define a valuation as a map \(v: K \to \mathbb{Z} \cup \{\infty\}\) that satisfies properties~\ref{prop:valuation-multiplication},~\ref{prop:valuation-addition}, and~\ref{prop:valuation-zero} of Theorem~\ref{thm:valuation-properties}. It can then be shown that for any valuation \(v\) the set \(P = \left\{ f \in K \,\middle|\, v(f) > 0 \right\}\) is a place whose valuation ring is given by \(\mathcal{O}_{P} = \left\{ f \in K \,\middle|\, v(f) \geq 0 \right\}\). It is therefore reasonable to say that places, valuations, and valuations rings are in some sense equivalent. They are all representations of the same underlying structure and one can freely move between each representation.
