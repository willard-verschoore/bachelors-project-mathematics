% -*- TeX-master: "../main.tex" -*-

\chapter{Background}%
\label{chap:background}

This chapter covers the theoretical background relevant to our work on solving the \(S\)-unit equation in function fields. We consider an arbitrary field \(k\) and we write \(k(x)\) to denote the field fractions of the polynomial ring \(k[x]\). The field \(k(x)\) is called the \textit{rational function field} over \(k\). We have the following definition.

\begin{definition}%
  \label{def:algebraic-function-field}
  An algebraic function field \(K / k\) of one variable is a finite field extension of the rational function field \(k(x)\).
\end{definition}

When we write \(K\), we always refer to an algebraic function field over \(k\) as given by Definition~\ref{def:algebraic-function-field}. Generally, we simply write ``function field'' instead of ``algebraic function field''.

In Section~\ref{sec:places-and-valuations}, we discuss places and valuations on function fields. Through the notion of divisors, places and valuations give rise to a decomposition of function field elements. Divisors and the related concept of Riemann-Roch spaces are the subject of Section~\ref{sec:divisors-and-riemann-roch-spaces}. Finally, in Section~\ref{sec:unit-equations} we explain how all this relates to \(S\)-units and the \(S\)-unit equation.

\section{Places and Valuations}%
\label{sec:places-and-valuations}

The related notions of places and valuations of function fields are fundamental to the formulation of the \(S\)-unit equation. We first define the concept of a valuation ring.

\begin{definition}%
  \label{def:valuation-ring}
  A valuation ring of the function field \(K / k\) is a ring \(\mathcal{O} \subseteq K\) with \(k \subset \mathcal{O} \subset K\) such that for every \(f \in K\), at least one of \(f\) or \(f^{-1}\) is in \(\mathcal{O}\).
\end{definition}

In the rational function field \(k(x)\), we can construct valuation rings by considering an irreducible monic polynomial \(p \in k[x]\) and taking
\[\mathcal{O}_{p} = \left\{ \frac{f}{g} \,\middle|\, f,g \in k[x] ,\, p \nmid g \right\}.\]
To see that this satisfies Definition~\ref{def:valuation-ring}, we note that every element of \(k(x)\) can be written as a reduced fraction \(f/g\) with \(f,g \in k[x]\) and \(p \nmid f\) or \(p \nmid g\). Therefore, we always have \(f/g \in \mathcal{O}_{p}\) or \(g/f \in \mathcal{O}_{p}\). We note that a different irreducible monic polynomial \(q \in k[x]\) gives rise to a different valuation ring \(\mathcal{O}_{q}\). For example, we have \(1/q \in \mathcal{O}_{p}\) but \(1/q \notin \mathcal{O}_{q}\).

The units of a valuation ring \(\mathcal{O}\) of the function field \(K\) are those \(f \in K\) such that \(f \in \mathcal{O}\) as well as \(f^{-1} \in \mathcal{O}\). For \(\mathcal{O}_{p}\) in the rational function field \(k(x)\) defined by an irreducible monic polynomial \(p \in k[x]\), this corresponds to the ratios of polynomials which do not contain a factor \(p\) in the numerator nor in the denominator. More precisely,
\[\mathcal{O}_{p}^{\times} = \left\{ \frac{f}{g} \,\middle|\, f,g \in k[x] ,\, p \nmid f ,\, p \nmid g \right\}.\]
This means that the ideal generated by \(p\) is given by \(p\mathcal{O}_{p} = \mathcal{O}_{p} \setminus \mathcal{O}_{p}^{\times}\). It follows that \(p\mathcal{O}_{p}\) is the unique maximal ideal of \(\mathcal{O}_{p}\) since any proper ideal \(I \subset \mathcal{O}_{p}\) cannot contain a unit.

This construction of a unique maximal ideal is not specific to valuation rings of the rational function field generated by an irreducible monic polynomial. The following theorem is a reformulation of Proposition 1.1.5 in~\cite{stichtenoth-2009-algebraic-function-fields}.

\begin{theorem}%
  \label{thm:place-is-maximal-ideal}
  A valuation ring \(\mathcal{O}\) of the function field \(K / k\) has a unique maximal ideal \(P = \mathcal{O} \setminus \mathcal{O}^{\times}\).
\end{theorem}

The unique maximal ideal \(P\) of a valuation ring \(\mathcal{O}\) is called a \textit{place}. The set of all places in the function field \(K\) is denoted by \(\mathbb{P}_{K}\). Given a place \(P\), we write \(\mathcal{O}_{P}\) for its corresponding valuation ring.

Since each place is a maximal ideal, the factor ring \(\mathcal{O}_{P} / P\) is a field. By Definition~\ref{def:valuation-ring} we have \(k \subset \mathcal{O}_{P}\), and thus \(\mathcal{O}_{P} / P\) extends the field of constants \(k\). This allows us to define the \textit{degree} of a place as the degree of this extension. More precisely, we write \(\deg{P} := [\mathcal{O}_{P} / P : k]\). % This is finite because it is bounded by the degree of the extension of \(K\) over \(k(x)\). We refer the reader to Proposition 1.1.15 in~\cite{stichtenoth-2009-algebraic-function-fields} for more details.

We saw that in the rational function field \(k(x)\) the irreducible monic polynomials correspond to distinct places and thus to distinct valuation rings. These are not the only places in \(k(x)\), however. There is one more valuation ring given by
\[\mathcal{O}_{\infty} = \left\{ \frac{f}{g} \,\middle|\, f,g \in k[x] ,\, \deg{f} \leq \deg{g} \right\},\]
which has the maximal ideal
\[P_{\infty} = \left\{ \frac{f}{g} \,\middle|\, f,g \in k[x] ,\, \deg{f} < \deg{g} \right\}.\]
The place \(P_{\infty}\) is referred to as the \textit{infinite place}. Note that \(P_{\infty} = x^{-1}\mathcal{O}_{\infty}\), so the infinite place is a principal ideal much like the other places.

This, again, is a property common to valuation rings and places of all function fields, not just those of the rational function field. The following theorem is a reformulation of Theorem 1.1.6 in~\cite{stichtenoth-2009-algebraic-function-fields}.

\begin{theorem}%
  \label{thm:place-is-principal-ideal}
  In a function field \(K / k\), every place \(P \in \mathbb{P}_{K}\) is a principal ideal of \(\mathcal{O}_{P}\). Moreover, for a \(p \in \mathcal{O}_{P}\) such that \(p\mathcal{O}_{p} = P\), we have that any \(f \in K\) can be uniquely written as \(f = p^{n}u\) with \(n \in \mathbb{Z}\) and \(u \in \mathcal{O}_{P}^{\times}\).
\end{theorem}

An element \(p\) which generates the place \(P\) is called a \textit{uniformizer} of \(P\). In the rational function field \(k(x)\), an irreducible monic polynomial is a uniformizer of its corresponding place.

We note that the integer \(n\) in the second part of Theorem~\ref{thm:place-is-principal-ideal} is unique for a given place \(P \in \mathbb{P}_{K}\) and element \(f \in K\). If \(p\) and \(q\) are distinct uniformizers of \(P\), then we have \(p = qs\) and \(q = pt\) for \(s,t \in \mathcal{O}_{P}\). In fact, \(s,t \in \mathcal{O}_{P}^{\times}\) since \(s = p/q\) and \(t = q/p\) are each other's inverses. It follows that if \(f = p^{n}u\) with \(u \in \mathcal{O}_{P}^{\times}\) then also \(f = q^{n}s^{n}u\) with \(s^{n}u \in \mathcal{O}_{P}^{\times}\). This justifies the following definition.

\begin{definition}%
  \label{def:valuation}
  For a place \(P \in \mathbb{P}_{K}\) with a uniformizer \(p\) we construct the function \(v_{P} : K \to \mathbb{Z} \cup \{\infty\}\) by considering for each non-zero \(f \in K\) the unique decomposition \(f = p^{n}u\) as in Theorem~\ref{thm:place-is-principal-ideal} and defining \(v_{P}(f) := n\). Additionally, we define \({v_{P}(0) := \infty}\).
\end{definition}

The function \(v_{P}\) is the \textit{valuation} associated with the place \(P\). As mentioned earlier, in the rational function field \(k(x)\), a finite place \(P\) has a uniformizer \(p\), which is an irreducible monic polynomial. For a rational function \(f \in k(x)\), the valuation \(v_{P}\) then counts the number of times \(p\) divides the numerator and denominator of \(f\), with occurrences in the numerator contributing positively and occurrences in the denominator contributing negatively. This motivates the following definition.

\begin{definition}%
  \label{def:zero-and-pole}
  For a function field element \(f \in K\), a place \(P \in \mathbb{P}_{K}\) is called a zero of \(f\) if \(v_{P}(f) > 0\). Alternatively, if \(v_{P}(f) < 0\) then we say that \(P\) is a pole of \(f\).
\end{definition}


Valuations as defined above have many nice properties. The following theorem is a reformulation of Theorem 1.1.13 in~\cite{stichtenoth-2009-algebraic-function-fields}.

\begin{theorem}%
  \label{thm:valuation-properties}
  In a function field \(K / k\) the valuation \(v_{P}\) for a place \(P \in \mathbb{P}_{K}\) as given by Definition~\ref{def:valuation} has the following properties:
  \begin{enumerate}[label = {(\arabic*)}]
    \item%
      \label{prop:valuation-multiplication}
      \(v_{P}(fg) = v_{P}(f) + v_{P}(g)\).

    \item%
      \label{prop:valuation-addition}
      \(v_{P}(f + g) \geq \min{\left\{v_{P}(f), v_{P}(g)\right\}}\).

    \item%
      \label{prop:valuation-zero}
      \(v_{P}(f) = \infty\) if and only if \(f = 0\).

    \item
      \(v_{P}(a) = 0\) for all \(a \in k\).

    \item
      \(v_{P}(f) = 1\) if and only if \(f\) is a uniformizer of \(P\).

    \item
      We have
      \begin{align*}
        \mathcal{O}_{P} = \left\{ f \in K \,\middle|\, v_{P}(f) \geq 0 \right\},       \\
        \mathcal{O}_{P}^{\times} = \left\{ f \in K \,\middle|\, v_{P}(f) = 0 \right\}, \\
        P = \left\{ f \in K \,\middle|\, v_{P}(f) > 0 \right\}.
      \end{align*}
  \end{enumerate}
\end{theorem}

We note that Definition~\ref{def:valuation} is not the only approach to defining valuations on function fields. It is also possible to define a valuation as a map \(v: K \to \mathbb{Z} \cup \{\infty\}\) that satisfies properties~\ref{prop:valuation-multiplication},~\ref{prop:valuation-addition}, and~\ref{prop:valuation-zero} of Theorem~\ref{thm:valuation-properties}. It can then be shown that for any valuation \(v\) the set \(P = \left\{ f \in K \,\middle|\, v(f) > 0 \right\}\) is a place whose valuation ring is given by \(\mathcal{O}_{P} = \left\{ f \in K \,\middle|\, v(f) \geq 0 \right\}\). It is therefore reasonable to say that places, valuations, and valuations rings are in some sense equivalent. They are all representations of the same underlying structure, and one can freely move between each representation.

\section{Divisors and Riemann-Roch Spaces}%
\label{sec:divisors-and-riemann-roch-spaces}

We have seen that in the rational function field \(k(x)\) finite places are associated with irreducible monic polynomials and the values of their valuations correspond to the multiplicity of these polynomials. In fact, for a rational function \(f \in k(x)\) there are a limited number of finite places \(P_{1}, P_{2}, \dots, P_{n}\) such that \(v_{P_{n}}(f) \neq 0\), and we can write
\[f = a p_{1}^{v_{P_{1}}(f)} p_{2}^{v_{P_{n}}(f)} \cdots p_{n}^{v_{P_{n}}(f)},\]
where \(a \in k\) and \(p_{i}\) is the irreducible monic polynomial corresponding to the place \(P_{i}\). One might say that \(f\) can be decomposed into a finite set of places, where each place is associated with an integer corresponding to the value of its valuation. This decomposition is unique for each element of \(k(x)\) up to a constant in \(k\).

Perhaps unsurprisingly, something similar holds true in any function field \(K\). For each element \(f \in K\) there are only a finite amount of places whose valuation is non-zero at \(f\). For a proof of this fact we refer the reader to Corollary 1.3.4 in~\cite{stichtenoth-2009-algebraic-function-fields}. We can therefore obtain a similar decomposition into pairs of places and integers as in the rational function field case. The following definition captures this concept.

\begin{definition}%
  \label{def:divisor}
  The divisor group \(\mathrm{Div}(K)\) of a function field \(K\) is the abelian group whose elements are formal sums of the places in \(\mathbb{P}_{K}\). In other words, for a \textit{divisor} \(D \in \mathrm{Div}(K)\) we have
  \[D = \sum_{P \in \mathbb{P}_{K}} n_{P} P,\]
  where \(n_{P} \in \mathbb{Z}\) and \(n_{P} \neq 0\) for finitely many \(P \in \mathbb{P}_{K}\). The addition of two divisors \({D = \sum n_{P}P}\) and \({D' = \sum n_{P}'P}\) is given by

  \[D + D' = \sum_{P \in \mathbb{P}_{K}} \left( n_{P} + n_{P}' \right) P.\]
  The \textit{degree} of a divisor \({D = \sum n_{P}P}\) is
  \[\deg{D} := \sum_{P \in \mathbb{P}_{K}} n_{P} \cdot \deg{P}.\]
  We associate with a non-zero element \(f \in K\) the \textit{principal divisor}
  \[(f) := \sum_{P \in \mathbb{P}_{K}} v_{P}(f) P.\]
\end{definition}

Note that the sum in Definition~\ref{def:divisor} is merely notation. One could equally well choose to write a divisor as a product
\[D = \prod_{P \in \mathbb{P}_{K}} P^{n_{P}}\]
and the group law as a multiplication
\[D \cdot D' = \prod_{P \in \mathbb{P}_{K}} P^{\left(n_{P} + n_{P}'\right)}.\]
This notation is perhaps more suggestive of the factorization into irreducible monic polynomials that we saw for elements of the rational function field \(k(x)\).

The following theorem, which corresponds to Theorem 1.4.11 in~\cite{stichtenoth-2009-algebraic-function-fields}, highlights an important property of principal divisors.

\begin{theorem}%
  \label{thm:principal-divisor-degree-zero}
  For a non-zero element \(f \in K\) we have that \(\deg{(f)} = 0\).
\end{theorem}

Theorem~\ref{thm:principal-divisor-degree-zero} is often referred to as the \textit{sum formula} or \textit{product formula}. It says that, when taking into account the degree of each place, the sum of all valuations of an element \(f \in K\) is zero. This property is fundamental to the generation of candidate solutions of the S-unit equation in our algorithm discussed in Chapter~\ref{chap:method}.

It is possible to construct a partial order on the divisor group \(\mathrm{Div}(K)\). Consider divisors \({D = \sum n_{P}P}\) and \({D' = \sum n_{P}'P}\), then we say that \(D \leq D'\) if and only if \(n_{P} \leq n_{P}'\) for every \(P \in \mathbb{P}_{K}\). This allows to express the following definition.

\begin{definition}%
  \label{def:riemann-roch-space}
  The Riemann-Roch space of a divisor \(D \in \mathrm{Div}(K)\) is defined as
  \[\mathcal{L}(D) := \left\{ f \in K \,\middle|\, (f) \geq -D \right\} \cup \{0\}.\]
\end{definition}

Riemann-Roch spaces are involved in many aspects of the theory of algebraic function fields. For example, they allow us to define the following important invariant of a function field.

\begin{definition}%
  \label{def:genus}
  The genus of a function field \(K / k\) is defined as
  \[g_{K} := \max{\left\{ \deg{D} - \dim{\mathcal{L}(D)} + 1 \,\middle|\, D \in \mathrm{Div}(K) \right\}}.\]
\end{definition}

In order to see that this maximum is well-defined we refer the reader to Proposition 1.4.14 in~\cite{stichtenoth-2009-algebraic-function-fields}. The genus of a function field notably appears the immensely important Riemann-Roch theorem. A discussion of the theorem is outside the scope of this work, however. We refer the reader to Theorem~1.5.15 in~\cite{stichtenoth-2009-algebraic-function-fields} and the surrounding discussion for more details.

There are many interesting results related to Riemann-Roch spaces. For our purposes, however, we only need the following two properties.

\begin{theorem}%
  \label{thm:riemann-roch-space-properties}
  Consider \(D \in \mathrm{Div}(K)\), then \(\mathcal{L}(D)\) has the following properties:
  \begin{enumerate}[label = {(\arabic*)}]
    \item%
      \label{prop:vector-space}
      \(\mathcal{L}(D)\) is a vector space over \(K\).

    \item%
      \label{prop:principal-dimension}
      If \(\deg{D} = 0\), then \(D\) is a principal divisor if and only if \(\dim{\mathcal{L}(D)} = 1\).
  \end{enumerate}
\end{theorem}

The first property corresponds to Lemma 1.4.6 in~\cite{stichtenoth-2009-algebraic-function-fields}, and for a proof of the second property we refer the reader to Corollary 1.4.12 in the same book.

Notably, Theorem~\ref{thm:riemann-roch-space-properties} implies that if we have a divisor \(D\) of degree zero we can construct its Riemann-Roch space \(\mathcal{L}(D)\) to find an element \(f \in K\) such that \((f) = D\). Namely, if \(\dim{\mathcal{L}(D)} = 1\) with the single basis vector \(g \in K\), then by Definition~\ref{def:riemann-roch-space} we have \((g) = -D\), and thus we can take \(f = g^{-1}\). This is an important tool when it comes to solving the \(S\)-unit equation on function fields.

\section{Unit Equations}%
\label{sec:unit-equations}

The theory discussed in Sections~\ref{sec:places-and-valuations} and~\ref{sec:divisors-and-riemann-roch-spaces} allows us to approach the problem central to our work: the \(S\)-unit equation. We first define the concept of an \(S\)-unit.

\begin{definition}
  Let \(S \subset \mathbb{P}_{K}\) be a finite set of places in a function field \(K / k\). The set of \(S\)-units is given by
  \[\left\{ f \in K \,\middle|\, f \in \mathcal{O}_{P}^{\times} \text{ for all } P \in \mathbb{P}_{K} \setminus S \right\}.\]
\end{definition}

In other words, an \(S\)-unit is a function field element which is a unit in every valuation ring, except possibly those given by the places in \(S\). Note that in the literature \(S\) is sometimes taken to be a set of valuations instead. In that case, an \(S\)-unit is an element whose valuation is zero everywhere, except possibly at the valuations in \(S\). It follows from Theorem~\ref{thm:valuation-properties} that these definitions are equivalent.

In the rational function field \(k(x)\), an element \(f\) is an \(S\)-unit if it is composed of only the irreducible monic polynomials corresponding to the finite places in \(S\). In general, an element \(f \in K\) is an \(S\)-unit if its divisor \((f)\) has non-zero coefficients only for the places in \(S\). Therefore, intuitively it might be said that an \(S\)-unit is an element that is ``composed'' of only the places in \(S\).

Given an \(S \subset \mathbb{P}_{K}\), we consider equations of the form
\[\gamma_{1} + \gamma_{2} + \gamma_{3} = 0,\]
where \(\gamma_{1}, \gamma_{2}, \gamma_{3} \in K\) are \(S\)-units. If we take \(\phi = -\gamma_{1} / \gamma_{3}\) and \(\psi = -\gamma_{2} / \gamma_{3}\) then \(\phi\) and \(\psi\) are \(S\)-units and
\[\phi + \psi = 1.\]
This is the standard \(S\)-unit equation.

In solving the \(S\)-unit equation, the following property of function field elements turns out to be of importance.

\begin{definition}%
  \label{def:height}
  The height of \(f \in K\) is defined as
  \[H(f) := - \sum_{P \in \mathbb{P}_{K}} \min{\left\{ 0, v_{P}(f) \cdot \deg{P} \right\}}.\]
\end{definition}

Essentially, this definition says that the height of \(f \in K\) is the magnitude of the sum of all negative coefficients in the divisor \((f)\). Theorem~\ref{thm:principal-divisor-degree-zero} says that the negative and positive coefficients in \((f)\) cancel out, so we can also write
\[H(f) = \sum_{P \in \mathbb{P}_{K}} \max{\left\{ 0, v_{P}(f) \cdot \deg{P} \right\}}.\]

The fundamental result that allows the \(S\)-unit equation to be solved is that the height of solutions is bounded. More precisely, for certain function fields we have the following important theorem.

\begin{theorem}%
  \label{thm:height-bound}
  Let \(k\) be a field of characteristic \(p \geq 0\) and \(K / k\) a function field. Consider a finite set \(S \subset \mathbb{P}_{K}\) and \(S\)-units \(\gamma_{1}, \gamma_{2}, \gamma_{3}\) satisfying
  \[\gamma_{1} + \gamma_{2} + \gamma_{3} = 0.\]
  Then we either have that \(\gamma_{1}/\gamma_{3}\) is an element of \(K^{p}\), or
  \[H\left(\frac{\gamma_{1}}{\gamma_{3}}\right) \leq 2g_{K} - 2 + \sum_{P \in S} \deg{P}.\]
  Here \(K^{p}\) is the set of \(p\)th powers in \(K\) for \(p > 0\) and \(K^{0} = k\).
\end{theorem}

This was first shown by Mason~\cite[Lemma 10]{mason-1984-diophantine-equations-over} for function fields \(K / k\) where \(k\) is algebraically closed. Later, Ga\'{a}l and Pohst~\cite{gaal-2006-diophantine-equations-over} expanded this result to function fields over finite fields, so \(k = \mathbb{F}_{q}\) where \(q = p^{n}\) and \(p\) is prime.
