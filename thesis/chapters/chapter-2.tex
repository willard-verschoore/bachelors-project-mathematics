% -*- TeX-master: "../main.tex" -*-

\chapter{Background}%
\label{chap:background}

Throughout this chapter we denote by \(k\) an arbitrary field. We write \(k(x)\) for the rational function field over \(k\), and we consider a finite algebraic extension \(K\) of \(k(x)\).\todo{Add an overview of upcoming sections.}

\section{Places and Valuations}%
\label{sec:places-and-valuations}

The related notions of places and valuations of function fields are fundamental to the formulation of the \(S\)-unit equation. We first define the concept of a valuation ring.

\begin{definition}%
  \label{def:valuation-ring}
  A valuation ring of the function field \(K\) is a ring \(\mathcal{O} \subset K\) such that for every \(f \in K\) we have that at least one of \(f\) or \(f^{-1}\) is in \(\mathcal{O}\).
\end{definition}

In the rational function field \(k(x)\) we can construct valuation rings by considering an irreducible sonic polynomial \(p \in k[x]\) and taking
\[\mathcal{O}_{p} = \left\{ \frac{f}{g} \,\middle|\, f,g \in k[x] ,\, p \nmid g \right\}.\]
To see that this satisfies Definition~\ref{def:valuation-ring} we note that every element of \(k(x)\) can be written as a reduced fraction \(f/g\) with \(f,g \in k[x]\) and \(p \nmid f\) or \(p \nmid g\). Therefore we always have \(f/g \in \mathcal{O}_{p}\) or \(g/f \in \mathcal{O}_{p}\). We note that a different irreducible monic polynomial \(q \in k[x]\) gives rise to a different valuation ring \(\mathcal{O}_{q}\). For example, we have \(1/q \in \mathcal{O}_{p}\) but \(1/q \notin \mathcal{O}_{q}\).

The units of a valuation ring \(\mathcal{O}\) of the function field \(K\) are those \(f \in K\) such that \(f \in \mathcal{O}\) as well as \(f^{-1} \in \mathcal{O}\). For \(\mathcal{O}_{p}\) in the rational function field \(k(x)\) defined by an irreducible monic polynomial \(p \in k[x]\) this corresponds to the ratios of polynomials which do not contain a factor \(p\) in the numerator nor in the denominator. More precisely,
\[\mathcal{O}_{p}^{\times} = \left\{ \frac{f}{g} \,\middle|\, f,g \in k[x] ,\, p \nmid f ,\, p \nmid g \right\}.\]
This means that the ideal generated by \(p\) is given by \((p) = \mathcal{O}_{p} \setminus \mathcal{O}_{p}^{\times}\). It follows that \((p)\) is the unique maximal ideal of \(\mathcal{O}_{p}\) since any proper ideal \(I \subset \mathcal{O}_{p}\) cannot contain a unit.

This property is not unique to valuation rings of the rational function field generated by an irreducible monic polynomial. The following theorem is a reformulation of Proposition 1.1.5 in~\cite{stichtenoth-2009-algebraic-function-fields}.

\begin{theorem}
  A valuation ring \(\mathcal{O}\) of the function field \(K\) has a unique maximal ideal \(P = \mathcal{O} \setminus \mathcal{O}^{\times}\).\todo{Perhaps expand this to include the properties from Theorem 1.1.6 in}
\end{theorem}
