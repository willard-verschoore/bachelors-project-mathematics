% -*- TeX-master: "../main.tex" -*-

\chapter{Introduction}%
\label{chap:introduction}

The equation \(x + y = 1\), where \(x\) and \(y\) are so-called \(S\)-units, plays an important role in algebraic number theory. An \(S\)-unit is a unit the ring of \(S\)-integers, which is a particular kind of subring of a field. Certain Diophantine equations, such as the Thue equation~\cite{thue-1909-uber-annaherungsweerte-algebraischer}, can be reduced to an \(S\)-unit equation. The \(S\)-unit equation also appears in matroid theory, where it corresponds to the problem of finding the fundamental elements of a partial field~\cite{zwam-2009-partial-fields-in}.

The \(S\)-unit equation has been extensively studied in number fields. The theory of linear forms in logarithms can be used to effectively determine an upper bound on the size of solutions to the \(S\)-unit equation~\cite{gyory-1979-on-the-number}. In theory this allows for an exhaustive search for all solutions, but in practice the bounds are often too big for this to be feasible. De Weger introduced a method for reducing the bounds to more reasonable size~\cite{weger-1989-algorithms-for-diophantine}, making use of the LLL-reduction algorithm developed by Lenstra, Lenstra, and Lov\'{a}sz~\cite{lenstra-1982-factoring-polynomials-with}. This method has since been built upon, for example by Alvarado et al.~\cite{alvarado-2021-a-robust-implementation}, whose algorithm is readily available in the computer algebra package SageMath~\cite{sagemath}.

The \(S\)-unit has also been studied in function fields, though to a lesser extent. Mason showed that a similar bound exists on the size of solutions to the \(S\)-unit in function fields over an algebraically closed base field~\cite{mason-1984-diophantine-equations-over}. Ga\'{a}l and Pohst extended this proof to function fields over a finite base field~\cite{gaal-2006-diophantine-equations-over}. This bound naturally leads to an algorithm for solving the \(S\)-unit equation in function fields based on an exhaustive search. However, no detailed description of this algorithm exists, and no implementation is publicly available.

We present a pseudocode description of the algorithm for solving \(S\)-unit equation in function fields as well as an implementation of the algorithm in SageMath. We provide the relevant background theory in Chapter~\ref{chap:background}. The algorithm is discussed in detail in Chapter~\ref{chap:method}. We apply the algorithm to a number of example problems in Chapter~\ref{chap:results}. We provide our conclusions and discuss potential future work in Chapter~\ref{chap:conclusion}.
