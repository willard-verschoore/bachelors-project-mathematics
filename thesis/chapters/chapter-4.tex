% -*- TeX-master: "../main.tex" -*-

\chapter{Results}%
\label{chap:results}

This is an introduction to the chapter.

\section{An Application in Matroid Theory}%
\label{sec:an-application-in-matroid-theory}

One area where the \(S\)-unit equation plays a role is in matroid theory, specifically in the study of \textit{partial fields}. A partial field \(\mathcal{P}\) is a generalization of the concept of a field and is given by a tuple \((R, G)\) where \(R\) is a ring and \(G \subseteq R^{\times}\) is a multiplicative group with \(-1 \in G\). We say that \(p \in \mathcal{P}\) whenever \(p \in G\) or \(p = 0\). Partial fields are often constructed by taking \(G\) to be the group generated by a finite set of units \(S \subset R^{\times}\), so \(G = \langle S \rangle\). For example, the \textit{dyadic partial field} is given by
\[\mathbb{D} := \left( \mathbb{Z} \left[ \frac{1}{2} \right], \left\langle -1, 2 \right\rangle \right).\]
The elements of \(\mathbb{D}\) are all \(p \in \mathbb{Z} \left[ \frac{1}{2} \right]\) where \(p = \pm 2^{n}\) for \(n \in \mathbb{Z}\). The thesis by van Zwam~\cite{zwam-2009-partial-fields-in} provides more thorough discussion of partial fields and the surrounding theory.

A partial field can be thought of as a field where the addition of two elements is not always defined. Pairs of elements that can be added are therefore of special interest. Notably, an element \(p \in \mathcal{P}\) such that \(p - 1 \in \mathcal{P}\) is called a \textit{fundament element}. Fundamental elements play an important role the study of partial fields and determining all fundamental elements is often necessary. Consider a partial field \(\mathcal{P} = (K, \langle S \rangle)\) where \(K\) is a function field and \(\langle S \rangle\) denotes the set of \(S\)-units generated by \(S \subset \mathbb{P}_{K}\), then the problem of finding the fundamental elements of \(\mathcal{P}\) is equivalent to finding all pairs \((f, g) \in K^{2}\) solving the \(S\)-unit equation. The algorithm of Chapter~\ref{chap:method} can therefore be applied to find the fundamental elements of certain partial fields.
