% -*- TeX-master: "../main.tex" -*-

\chapter{Results}%
\label{chap:results}

In this chapter we demonstrate the algorithm by applying it to some example problems. Section~\ref{sec:examples} contains a simple example for each type of function field our algorithm can be applied to: one over an algebraically closed base field and one over a finite base field. In Section~\ref{sec:an-application-in-matroid-theory}, we discuss an application of the algorithm in the field matroid theory.

\section{Examples}%
\label{sec:examples}

To get an impression of the functioning of the Algorithm of Chapter~\ref{chap:method} we consider two examples. First, let \(K = \bar{\mathbb{Q}}(x)\) and
\[S = \left\{ x, x^{-1}, x - 1, x + 2 \right\}.\]
Here we use irreducible polynomials to refer to the corresponding places in \(\mathbb{P}_{K}\) and \(x^{-1}\) denotes the infinite place. From Theorem~\ref{thm:height-bound} we obtain the height bound
\[N = 2g_{K} - 2 + \sum_{P \in S} \deg{P} = 0 - 2 + 4 = 2,\]
so we are interested in \(S\)-units \(f \in \bar{\mathbb{Q}}(x)\) with \(H(f) = 1\) or \(H(f) = 2\). Our implementation of Algorithm~\ref{alg:bounded-s-units} finds \(54\) such \(S\)-units. By considering each pair of the same height and solving the resulting system of equations as discussed in Section~\ref{sec:the-algorithm} we find all solutions to the \(S\)-unit equation. They are given by
\[x - (x - 1) = 1, \quad \frac{x + 2}{2} - \frac{x}{2} = 1, \quad \frac{x + 2}{3} - \frac{x - 1}{3} = 1, \quad \frac{3x}{x + 2} - \frac{2(x - 1)}{x + 2} = 1.\]
Note that each of these solutions is a representative of a group of six solutions as discussed in Section~\ref{sec:complexity-and-performance}.

Let us now consider another example where we take \(K = \mathbb{F}_{5}(x)\) and
\[S = \left\{ x, x^{-1}, x + \bar{2}, x^{2} + x + \bar{1} \right\}.\]
In this case the height bound becomes
\[N = 2g_{K} - 2 + \sum_{P \in S} \deg{P} = 0 - 2 + 5 = 3\]
and Algorithm~\ref{alg:bounded-s-units} finds \(72\) \(S\)-units satisfying this bound. While there are more candidates compared to the previous example, there are only two solutions to the \(S\)-unit equation. They are given by
\[\frac{\bar{3}}{x}  + \frac{x + \bar{2}}{x} = \bar{1}, \quad \frac{\bar{3}(x^{2} + x + \bar{1})}{x^{2}} + \frac{\bar{3}{(x + \bar{2})}^{2}}{x^{2}} = \bar{1}.\]
If we include the place corresponding to \(x - \bar{1}\) in \(S\), the height bound increases by one. The number of bounded \(S\)-units grows much more, however. Algorithm~\ref{alg:bounded-s-units} produces \(658\) candidates and we find \(13\) solutions. This illustrates the outsize effect the number of places in \(S\) has on the number of (candidate) solutions.

\section{An Application in Matroid Theory}%
\label{sec:an-application-in-matroid-theory}

One area where the \(S\)-unit equation plays a role is in matroid theory, specifically in the study of \textit{partial fields}. A partial field \(\mathcal{P}\) is a generalization of the concept of a field and is given by a tuple \((R, G)\) where \(R\) is a ring and \(G \subseteq R^{\times}\) is a multiplicative group with \(-1 \in G\). We say that \(p \in \mathcal{P}\) whenever \(p \in G\) or \(p = 0\). Partial fields are often constructed by taking \(G\) to be the group generated by a finite set of units \(S \subset R^{\times}\), so \(G = \langle S \rangle\). For example, the \textit{dyadic partial field} is given by
\[\mathbb{D} := \left( \mathbb{Z} \left[ \frac{1}{2} \right], \left\langle -1, 2 \right\rangle \right).\]
The elements of \(\mathbb{D}\) are all \(p \in \mathbb{Z} \left[ \frac{1}{2} \right]\) where \(p = \pm 2^{n}\) for \(n \in \mathbb{Z}\). The thesis by van Zwam~\cite{zwam-2009-partial-fields-in} provides more thorough discussion of partial fields and the surrounding theory.

A partial field can be thought of as a field where the addition of two elements is not always defined. Pairs of elements that can be added are therefore of special interest. Notably, an element \(p \in \mathcal{P}\) such that \(p - 1 \in \mathcal{P}\) is called a \textit{fundament element}. Fundamental elements play an important role the study of partial fields and determining all fundamental elements is often necessary. If we consider a partial field \(\mathcal{P} = (K, \langle S \rangle)\) where \(K\) is a function field and \(\langle S \rangle\) denotes the set of \(S\)-units generated by \(S \subset \mathbb{P}_{K}\), then the problem of finding the fundamental elements of \(\mathcal{P}\) is equivalent to finding all pairs \((f, g) \in K^{2}\) solving the \(S\)-unit equation. The algorithm of Chapter~\ref{chap:method} can therefore be applied to find the fundamental elements of certain partial fields.

For an example application of our algorithm in finding fundamental elements we turn to another generalization of fields called \textit{pastures}. Pastures are a more general category than partial fields; every partial field is a pasture but not every pasture is a partial field. We do not discuss pastures in detail here, for that we refer the reader to the work of Baker and Lorscheid~\cite{baker-2020-foundations-of-matroids}.  For our purposes it is sufficient to know that a pasture can be defined in terms of a partial field and a set of equations of the form
\[p + q = 1\]
listing pairs of fundamental elements of the partial field.\todo{Is this true for all or only some pastures?} If all fundamental elements are included in the definition of the pasture, it is itself a partial field.

Appendix A in the work by Baker et al.~\cite{baker-2023-foundations-of-matroids} contains a number of pastures of this form. For example, section A.3.31 discusses a pasture constructed over the partial field
\[\mathbb{F}_{1}^{\pm}(x) = \left( \mathbb{Z} \left[ x, x^{-1} \right], \left\langle -1, x, x^{-1}, x -1, 2x - 1, x^{2} + x - 1 \right\rangle \right)\]
with the equations
\begin{equation*}
  \def\arraystretch{1.5}
  \begin{array}{cc}
    x - (x - 1) = 1,                                                                  & \frac{2x - 1}{x} - \frac{x - 1}{x} = 1,                                    \\
    \frac{x^{2}}{x^{2} + x - 1} + \frac{x - 1}{x^{2} + x - 1} = 1,                    & \frac{2x - 1}{x} - \frac{x - 1}{x} = 1,                                    \\
    \frac{2x - 1}{x^{2} + x - 1} + \frac{x(x - 1)}{x^{2} + x - 1} = 1,                & \frac{x(2x - 1)}{x^{2} + x - 1} - \frac{{(x - 1)}^{2}}{x^{2} + x - 1} = 1, \\
    \frac{x^{3}}{(x - 1)(x^{2} + x - 1)} - \frac{2x - 1}{(x - 1)(x^{2} + x - 1)} = 1. &
  \end{array}
\end{equation*}
We can determine whether these are all fundamental elements of the partial field \(\mathbb{F}_{1}^{\pm}(x)\) by considering the \(S\)-unit equation over the function field \(K = \bar{\mathbb{Q}}(x)\) with
\[S = \left\{ x, x^{-1}, x -1, 2x - 1, x^{2} + x - 1 \right\},\]
where we use irreducible polynomials to refer the corresponding places in \(\mathbb{P}_{K}\). Any solution involving a constant other than \(\pm 1\) is not valid in the partial field and can therefore be dismissed. With this setup, the algorithm of Chapter~\ref{chap:method} yields all solutions corresponding to the pasture's equations as well as one additional solution given by
\[\frac{x(x^{2} + x - 1)}{{(x - 1)}^{3}} - \frac{{(2x - 1)}^{2}}{{(x - 1)}^{3}} = 1.\]
We can thus conclude that the pasture is not a partial field.
