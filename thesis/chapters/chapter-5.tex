% -*- TeX-master: "../main.tex" -*-

\chapter{Conclusion}%
\label{chap:conclusion}

We have implemented an algorithm for solving the \(S\)-unit equation in function fields over an algebraically closed or finite base field. The algorithm uses the theory of integer partitions to determine all possible divisors of \(S\)-units below the height bound of Theorem~\ref{thm:height-bound}. The \(S\)-units themselves are then constructed from the divisors by considering the basis of their Riemann-Roch space. The existence of such an algorithm was already discussed by Mason in~\cite{mason-1984-diophantine-equations-over}, but no detailed discussion or implementation was publicly available up to now.

We also derived an upper and lower bound on the number of candidate solutions considered by algorithm. Both bounds grow rapidly with the number of places in the set \(S\). As a result, the algorithm is impractical when dealing with \(S\)-units equations over a large set \(S\). However, the algorithm is still useful since problems with relatively few places are already often impossible to solve by hand. An example of such a problem in matroid theory was provided in Section~\ref{sec:an-application-in-matroid-theory}, and Appendix A in~\cite{baker-2024-foundations-of-matroids} contains many more.

Future work might focus on improving the performance of the algorithm by reducing the height bound. In number fields this has been achieved through the use of LLL-reduction and sieve techniques. Similar methods could explored for use in the function field case. Even without such improvements, the algorithm can be applied to solving certain Diophantine equations. It could for example be adopted for use in the method for solving the Thue equation discussed by Mason in~\cite{mason-1984-diophantine-equations-over} and Ga\'{a}l and Pohst in~\cite{gaal-2006-diophantine-equations-over, gaal-2006-diophantine-equations-overa}.
