% -*- TeX-master: "main.tex" -*-

\documentclass[a4paper]{report}

\usepackage{amsmath}
\usepackage{amssymb}
\usepackage{amsthm}

\usepackage{enumitem}

% Bibliography.
\usepackage{biblatex}
\addbibresource{bibliography.bib}

% To-do items.
\usepackage[textwidth=4cm]{todonotes}

\newtheorem{theorem}{Theorem}
\newtheorem{corollary}{Corollary}[theorem]
\newtheorem{lemma}[theorem]{Lemma}

\theoremstyle{definition}
\newtheorem{definition}{Definition}

\theoremstyle{remark}
\newtheorem*{remark}{Remark}

\usepackage{algorithm}
\usepackage{algpseudocode}

% Title page.
\usepackage[english]{babel}
\usepackage{styles/rugscriptie}

\faculty{fse}
\thesistype{Bachelor's Project Mathematics}
\title{Solving the S-Unit Equation in Function Fields}
\author{W.A.~Verschoore de la Houssaije}
\supervisor{O.~Lorscheid}
\supervisor{J.S.~M\"{u}ller}


\begin{document}

\maketitle

\begin{abstract}
  In algebraic number theory, the equation \(x + y = 1\), where \(x\) and \(y\) are units in a certain subring of a field, plays an important role in solving various Diophantine equations as well as in matroid theory. The type of subring containing \(x\) and \(y\) is known as a ring of \(S\)-integers, and the equation is called the \(S\)-unit equation. Much work has been done on developing efficient algorithms for solving the S-unit equation in number fields. Implementations of these algorithms are readily available, for example in the computer algebra package SageMath. While an algorithm for solving the S-unit equation in certain function fields is known, no implementation is publicly available.

  We present an implementation of the algorithm for solving the S-unit equation in function fields. We discuss the theory underlying the algorithm and demonstrate its effectiveness through a number of examples. Notably, our implementation has already proven useful in research in the field of matroid theory.
\end{abstract}

% -*- TeX-master: "../main.tex" -*-

\chapter{Introduction}%
\label{chap:introduction}

This is an introduction.


% -*- TeX-master: "../main.tex" -*-

\chapter{Background}%
\label{chap:background}

This chapter covers the theoretical background relevant to our work on solving the \(S\)-unit equation in function fields. We consider an arbitrary field \(k\) and we write \(k(x)\) to denote the field fractions of the polynomial ring \(k[x]\). The field \(k(x)\) is called the \textit{rational function field} over \(k\). We have the following definition.

\begin{definition}%
  \label{def:algebraic-function-field}
  An algebraic function field \(K / k\) of one variable is a finite field extension of the rational function field \(k(x)\).
\end{definition}

When we write \(K\), we always refer to an algebraic function field over \(k\) as given by Definition~\ref{def:algebraic-function-field}. Generally, we simply write ``function field'' instead of ``algebraic function field''.

In Section~\ref{sec:places-and-valuations}, we discuss places and valuations on function fields. Through the notion of divisors, places and valuations give rise to a decomposition of function field elements. Divisors and the related concept of Riemann-Roch spaces are the subject of Section~\ref{sec:divisors-and-riemann-roch-spaces}. Finally, in Section~\ref{sec:unit-equations} we explain how all this relates to \(S\)-units and the \(S\)-unit equation.

\section{Places and Valuations}%
\label{sec:places-and-valuations}

The related notions of places and valuations of function fields are fundamental to the formulation of the \(S\)-unit equation. We first define the concept of a valuation ring.

\begin{definition}%
  \label{def:valuation-ring}
  A valuation ring of the function field \(K / k\) is a ring \(\mathcal{O} \subseteq K\) with \(k \subset \mathcal{O} \subset K\) such that for every \(f \in K\), at least one of \(f\) or \(f^{-1}\) is in \(\mathcal{O}\).
\end{definition}

In the rational function field \(k(x)\), we can construct valuation rings by considering an irreducible monic polynomial \(p \in k[x]\) and taking
\[\mathcal{O}_{p} = \left\{ \frac{f}{g} \,\middle|\, f,g \in k[x] ,\, p \nmid g \right\}.\]
To see that this satisfies Definition~\ref{def:valuation-ring}, we note that every element of \(k(x)\) can be written as a reduced fraction \(f/g\) with \(f,g \in k[x]\) and \(p \nmid f\) or \(p \nmid g\). Therefore, we always have \(f/g \in \mathcal{O}_{p}\) or \(g/f \in \mathcal{O}_{p}\). We note that a different irreducible monic polynomial \(q \in k[x]\) gives rise to a different valuation ring \(\mathcal{O}_{q}\). For example, we have \(1/q \in \mathcal{O}_{p}\) but \(1/q \notin \mathcal{O}_{q}\).

The units of a valuation ring \(\mathcal{O}\) of the function field \(K\) are those \(f \in K\) such that \(f \in \mathcal{O}\) as well as \(f^{-1} \in \mathcal{O}\). For \(\mathcal{O}_{p}\) in the rational function field \(k(x)\) defined by an irreducible monic polynomial \(p \in k[x]\), this corresponds to the ratios of polynomials which do not contain a factor \(p\) in the numerator nor in the denominator. More precisely,
\[\mathcal{O}_{p}^{\times} = \left\{ \frac{f}{g} \,\middle|\, f,g \in k[x] ,\, p \nmid f ,\, p \nmid g \right\}.\]
This means that the ideal generated by \(p\) is given by \(p\mathcal{O}_{p} = \mathcal{O}_{p} \setminus \mathcal{O}_{p}^{\times}\). It follows that \(p\mathcal{O}_{p}\) is the unique maximal ideal of \(\mathcal{O}_{p}\) since any proper ideal \(I \subset \mathcal{O}_{p}\) cannot contain a unit.

This construction of a unique maximal ideal is not specific to valuation rings of the rational function field generated by an irreducible monic polynomial. The following theorem is a reformulation of Proposition 1.1.5 in~\cite{stichtenoth-2009-algebraic-function-fields}.

\begin{theorem}%
  \label{thm:place-is-maximal-ideal}
  A valuation ring \(\mathcal{O}\) of the function field \(K / k\) has a unique maximal ideal \(P = \mathcal{O} \setminus \mathcal{O}^{\times}\).
\end{theorem}

The unique maximal ideal \(P\) of a valuation ring \(\mathcal{O}\) is called a \textit{place}. The set of all places in the function field \(K\) is denoted by \(\mathbb{P}_{K}\). Given a place \(P\), we write \(\mathcal{O}_{P}\) for its corresponding valuation ring.

Since each place is a maximal ideal, the factor ring \(\mathcal{O}_{P} / P\) is a field. By Definition~\ref{def:valuation-ring} we have \(k \subset \mathcal{O}_{P}\), and thus \(\mathcal{O}_{P} / P\) extends the field of constants \(k\). This allows us to define the \textit{degree} of a place as the degree of this extension. More precisely, we write \(\deg{P} := [\mathcal{O}_{P} / P : k]\). % This is finite because it is bounded by the degree of the extension of \(K\) over \(k(x)\). We refer the reader to Proposition 1.1.15 in~\cite{stichtenoth-2009-algebraic-function-fields} for more details.

We saw that in the rational function field \(k(x)\) the irreducible monic polynomials correspond to distinct places and thus to distinct valuation rings. These are not the only places in \(k(x)\), however. There is one more valuation ring given by
\[\mathcal{O}_{\infty} = \left\{ \frac{f}{g} \,\middle|\, f,g \in k[x] ,\, \deg{f} \leq \deg{g} \right\},\]
which has the maximal ideal
\[P_{\infty} = \left\{ \frac{f}{g} \,\middle|\, f,g \in k[x] ,\, \deg{f} < \deg{g} \right\}.\]
The place \(P_{\infty}\) is referred to as the \textit{infinite place}. Note that \(P_{\infty} = x^{-1}\mathcal{O}_{\infty}\), so the infinite place is a principal ideal much like the other places.

This, again, is a property common to valuation rings and places of all function fields, not just those of the rational function field. The following theorem is a reformulation of Theorem 1.1.6 in~\cite{stichtenoth-2009-algebraic-function-fields}.

\begin{theorem}%
  \label{thm:place-is-principal-ideal}
  In a function field \(K / k\), every place \(P \in \mathbb{P}_{K}\) is a principal ideal of \(\mathcal{O}_{P}\). Moreover, for a \(t \in \mathcal{O}_{P}\) such that \(t\mathcal{O}_{p} = P\), we have that any \(f \in K\) can be uniquely written as \(f = t^{n}u\) with \(n \in \mathbb{Z}\) and \(u \in \mathcal{O}_{P}^{\times}\).
\end{theorem}

An element \(t\) which generates the place \(P\) is called a \textit{uniformizer} of \(P\). In the rational function field \(k(x)\), an irreducible monic polynomial is a uniformizer of its corresponding place.

We note that the integer \(n\) in the second part of Theorem~\ref{thm:place-is-principal-ideal} is unique for a given place \(P \in \mathbb{P}_{K}\) and element \(f \in K\). If \(t\) and \(t'\) are distinct uniformizers of \(P\), then we have \(t = t'w'\) and \(t' = tw\) for \(w,w' \in \mathcal{O}_{P}\). In fact, \(w,w' \in \mathcal{O}_{P}^{\times}\) since \(w = t'/t\) and \(w' = t/t'\) are each other's inverses. It follows that if \(f = t^{n}u\) with \(u \in \mathcal{O}_{P}^{\times}\) then also \(f = {t'}^{n}{w'}^{n}u\) with \({w'}^{n}u \in \mathcal{O}_{P}^{\times}\). This justifies the following definition.

\begin{definition}%
  \label{def:valuation}
  For a place \(P \in \mathbb{P}_{K}\) with a uniformizer \(t\) we construct the function \(v_{P} : K \to \mathbb{Z} \cup \{\infty\}\) by considering for each non-zero \(f \in K\) the unique decomposition \(f = t^{n}u\) as in Theorem~\ref{thm:place-is-principal-ideal} and defining \(v_{P}(f) := n\). Additionally, we define \({v_{P}(0) := \infty}\).
\end{definition}

The function \(v_{P}\) is the \textit{valuation} associated with the place \(P\). As mentioned earlier, in the rational function field \(k(x)\), a finite place \(P\) has a uniformizer \(t\), which is an irreducible monic polynomial. For a rational function \(f \in k(x)\), the valuation \(v_{P}\) then counts the number of times \(t\) divides the numerator and denominator of \(f\), with occurrences in the numerator contributing positively and occurrences in the denominator contributing negatively. This motivates the following definition.

\begin{definition}%
  \label{def:zero-and-pole}
  For a function field element \(f \in K\), a place \(P \in \mathbb{P}_{K}\) is called a zero of \(f\) if \(v_{P}(f) > 0\). Alternatively, if \(v_{P}(f) < 0\) then we say that \(P\) is a pole of \(f\).
\end{definition}


Valuations as defined above have many nice properties. The following theorem is a reformulation of Theorem 1.1.13 in~\cite{stichtenoth-2009-algebraic-function-fields}.

\begin{theorem}%
  \label{thm:valuation-properties}
  In a function field \(K / k\) the valuation \(v_{P}\) for a place \(P \in \mathbb{P}_{K}\) as given by Definition~\ref{def:valuation} has the following properties:
  \begin{enumerate}[label = {(\arabic*)}]
    \item%
      \label{prop:valuation-multiplication}
      \(v_{P}(fg) = v_{P}(f) + v_{P}(g)\).

    \item%
      \label{prop:valuation-addition}
      \(v_{P}(f + g) \geq \min{\left\{v_{P}(f), v_{P}(g)\right\}}\).

    \item%
      \label{prop:valuation-zero}
      \(v_{P}(f) = \infty\) if and only if \(f = 0\).

    \item
      \(v_{P}(a) = 0\) for all \(a \in k\).

    \item
      \(v_{P}(f) = 1\) if and only if \(f\) is a uniformizer of \(P\).

    \item
      We have
      \begin{align*}
        \mathcal{O}_{P} = \left\{ f \in K \,\middle|\, v_{P}(f) \geq 0 \right\},       \\
        \mathcal{O}_{P}^{\times} = \left\{ f \in K \,\middle|\, v_{P}(f) = 0 \right\}, \\
        P = \left\{ f \in K \,\middle|\, v_{P}(f) > 0 \right\}.
      \end{align*}
  \end{enumerate}
\end{theorem}

We note that Definition~\ref{def:valuation} is not the only approach to defining valuations on function fields. It is also possible to define a valuation as a map \(v: K \to \mathbb{Z} \cup \{\infty\}\) that satisfies properties~\ref{prop:valuation-multiplication},~\ref{prop:valuation-addition}, and~\ref{prop:valuation-zero} of Theorem~\ref{thm:valuation-properties}. It can then be shown that for any valuation \(v\) the set \(P = \left\{ f \in K \,\middle|\, v(f) > 0 \right\}\) is a place whose valuation ring is given by \(\mathcal{O}_{P} = \left\{ f \in K \,\middle|\, v(f) \geq 0 \right\}\). It is therefore reasonable to say that places, valuations, and valuations rings are in some sense equivalent. They are all representations of the same underlying structure, and one can freely move between each representation.

\section{Divisors and Riemann-Roch Spaces}%
\label{sec:divisors-and-riemann-roch-spaces}

We have seen that in the rational function field \(k(x)\) finite places are associated with irreducible monic polynomials and the values of their valuations correspond to the multiplicity of these polynomials. In fact, for a rational function \(f \in k(x)\) there are a limited number of finite places \(P_{1}, P_{2}, \dots, P_{n}\) such that \(v_{P_{n}}(f) \neq 0\), and we can write
\[f = a p_{1}^{v_{P_{1}}(f)} p_{2}^{v_{P_{n}}(f)} \cdots p_{n}^{v_{P_{n}}(f)},\]
where \(a \in k\) and \(p_{i}\) is the irreducible monic polynomial corresponding to the place \(P_{i}\). One might say that \(f\) can be decomposed into a finite set of places, where each place is associated with an integer corresponding to the value of its valuation. This decomposition is unique for each element of \(k(x)\) up to a constant in \(k\).

Perhaps unsurprisingly, something similar holds true in any function field \(K\). For each element \(f \in K\) there are only a finite amount of places whose valuation is non-zero at \(f\). For a proof of this fact we refer the reader to Corollary 1.3.4 in~\cite{stichtenoth-2009-algebraic-function-fields}. We can therefore obtain a similar decomposition into pairs of places and integers as in the rational function field case. The following definition captures this concept.

\begin{definition}%
  \label{def:divisor}
  The divisor group \(\mathrm{Div}(K)\) of a function field \(K\) is the abelian group whose elements are finite formal sums of the places in \(\mathbb{P}_{K}\). In other words, for a \textit{divisor} \(D \in \mathrm{Div}(K)\) we have
  \[D = \sum_{P \in \mathbb{P}_{K}} n_{P} P,\]
  where \(n_{P} \in \mathbb{Z}\) and \(n_{P} \neq 0\) for finitely many \(P \in \mathbb{P}_{K}\). The addition of two divisors \({D = \sum n_{P}P}\) and \({D' = \sum n_{P}'P}\) is given by

  \[D + D' = \sum_{P \in \mathbb{P}_{K}} \left( n_{P} + n_{P}' \right) P.\]
  The \textit{degree} of a divisor \({D = \sum n_{P}P}\) is
  \[\deg{D} := \sum_{P \in \mathbb{P}_{K}} n_{P} \cdot \deg{P}.\]
  We associate with a non-zero element \(f \in K\) the \textit{principal divisor}
  \[(f) := \sum_{P \in \mathbb{P}_{K}} v_{P}(f) P.\]
\end{definition}

Note that the sum in Definition~\ref{def:divisor} is merely notation. One could equally well choose to write a divisor as a product
\[D = \prod_{P \in \mathbb{P}_{K}} P^{n_{P}}\]
and the group law as a multiplication
\[D \cdot D' = \prod_{P \in \mathbb{P}_{K}} P^{\left(n_{P} + n_{P}'\right)}.\]
This notation is perhaps more suggestive of the factorization into irreducible monic polynomials that we saw for elements of the rational function field \(k(x)\).

The following theorem, which corresponds to Theorem 1.4.11 in~\cite{stichtenoth-2009-algebraic-function-fields}, highlights an important property of principal divisors.

\begin{theorem}%
  \label{thm:principal-divisor-degree-zero}
  For a non-zero element \(f \in K\) we have that \(\deg{(f)} = 0\).
\end{theorem}

Theorem~\ref{thm:principal-divisor-degree-zero} is often referred to as the \textit{sum formula} or \textit{product formula}. It says that, when taking into account the degree of each place, the sum of all valuations of an element \(f \in K\) is zero. This property is fundamental to the generation of candidate solutions of the S-unit equation in our algorithm discussed in Chapter~\ref{chap:method}.

It is possible to construct a partial order on the divisor group \(\mathrm{Div}(K)\). Consider divisors \({D = \sum n_{P}P}\) and \({D' = \sum n_{P}'P}\), then we say that \(D \leq D'\) if and only if \(n_{P} \leq n_{P}'\) for every \(P \in \mathbb{P}_{K}\). This allows to express the following definition.

\begin{definition}%
  \label{def:riemann-roch-space}
  The Riemann-Roch space of a divisor \(D \in \mathrm{Div}(K)\) is defined as
  \[\mathcal{L}(D) := \left\{ f \in K^{\times} \,\middle|\, (f) \geq -D \right\} \cup \{0\}.\]
\end{definition}

Riemann-Roch spaces are involved in many aspects of the theory of algebraic function fields. For example, they allow us to define the following important invariant of a function field.

\begin{definition}%
  \label{def:genus}
  The genus of a function field \(K / k\) is defined as
  \[g_{K} := \max{\left\{ \deg{D} - \dim{\mathcal{L}(D)} + 1 \,\middle|\, D \in \mathrm{Div}(K) \right\}}.\]
\end{definition}

In order to see that this maximum is well-defined we refer the reader to Proposition 1.4.14 in~\cite{stichtenoth-2009-algebraic-function-fields}. The genus of a function field notably appears the immensely important Riemann-Roch theorem.

\begin{theorem}
  There exists a divisor \(W \in \mathrm{Div}(K)\) (called a \textit{canonical divisor}) such that for every other divisor \(D \in \mathrm{Div}(K)\), the equality
  \[\dim{\mathcal{L}(D)} - \dim{\mathcal{L}(W - D)} = \deg{D} - g_{K} + 1\]
  holds. Moreover, the canonical divisor \(W\) is unique up to addition by a principal divisor.
\end{theorem}

A proper discussion of this theorem is outside the scope of this work. For more information we refer the reader to Section~1.5 in~\cite{stichtenoth-2009-algebraic-function-fields}, which covers the Riemann-Roch Theorem in detail.

There are many interesting results related to Riemann-Roch spaces. For our purposes, however, we only need the following two properties.

\begin{theorem}%
  \label{thm:riemann-roch-space-properties}
  Consider \(D \in \mathrm{Div}(K)\), then \(\mathcal{L}(D)\) has the following properties:
  \begin{enumerate}[label = {(\arabic*)}]
    \item%
      \label{prop:vector-space}
      \(\mathcal{L}(D)\) is a vector space over \(K\).

    \item%
      \label{prop:principal-dimension}
      If \(\deg{D} = 0\), then \(D\) is a principal divisor if and only if \(\dim{\mathcal{L}(D)} = 1\).
  \end{enumerate}
\end{theorem}

The first property corresponds to Lemma 1.4.6 in~\cite{stichtenoth-2009-algebraic-function-fields}, and for a proof of the second property we refer the reader to Corollary 1.4.12 in the same book.

Notably, Theorem~\ref{thm:riemann-roch-space-properties} implies that if we have a divisor \(D\) of degree zero we can construct its Riemann-Roch space \(\mathcal{L}(D)\) to find an element \(f \in K\) such that \((f) = D\). Namely, if \(\dim{\mathcal{L}(D)} = 1\) with the single basis vector \(g \in K\), then by Definition~\ref{def:riemann-roch-space} we have \((g) = -D\), and thus we can take \(f = g^{-1}\). This is an important tool when it comes to solving the \(S\)-unit equation on function fields.

Some computer algebra systems have built-in functionality for determining whether a divisor is principal and can find a generator if it is. For example, Magma~\cite{bosma-1997-the-magma-algebra} has an \textsc{IsPrincipal} function\footnote{\href{https://magma.maths.usyd.edu.au/magma/handbook/text/1393\#15776}{\texttt{magma.maths.usyd.edu.au/magma/handbook/text/1393\#15776}}} which does exactly this. The documentation of this function states that it ``uses the Riemann-Roch space of [the divisor]''. Presumably the Riemann-Roch space is used in a process similar to that described above.

\section{Unit Equations}%
\label{sec:unit-equations}

The theory discussed in Sections~\ref{sec:places-and-valuations} and~\ref{sec:divisors-and-riemann-roch-spaces} allows us to approach the problem central to our work: the \(S\)-unit equation. We first define the concept of an \(S\)-unit.

\begin{definition}
  Let \(S \subset \mathbb{P}_{K}\) be a finite set of places in a function field \(K / k\). The set of \(S\)-units is given by
  \[\left\{ f \in K \,\middle|\, f \in \mathcal{O}_{P}^{\times} \text{ for all } P \in \mathbb{P}_{K} \setminus S \right\}.\]
\end{definition}

In other words, an \(S\)-unit is a function field element which is a unit in every valuation ring, except possibly those given by the places in \(S\). Note that in the literature \(S\) is sometimes taken to be a set of valuations instead. In that case, an \(S\)-unit is an element whose valuation is zero everywhere, except possibly at the valuations in \(S\). It follows from Theorem~\ref{thm:valuation-properties} that these definitions are equivalent.

In the rational function field \(k(x)\) with a set \(S \in \mathbb{P}_{K}\) containing the infinite place, an element \(f\) is an \(S\)-unit if it is composed of only the irreducible monic polynomials corresponding to the finite places in \(S\). In general, an element \(f \in K\) is an \(S\)-unit if its divisor \((f)\) has non-zero coefficients only for the places in \(S\). Therefore, intuitively it might be said that an \(S\)-unit is an element that is ``composed'' of only the places in \(S\).

Given an \(S \subset \mathbb{P}_{K}\), we consider equations of the form
\[\gamma_{1} + \gamma_{2} + \gamma_{3} = 0,\]
where \(\gamma_{1}, \gamma_{2}, \gamma_{3} \in K\) are \(S\)-units. If we take \(\phi = -\gamma_{1} / \gamma_{3}\) and \(\psi = -\gamma_{2} / \gamma_{3}\) then \(\phi\) and \(\psi\) are \(S\)-units and
\[\phi + \psi = 1.\]
This is the standard \(S\)-unit equation.

In solving the \(S\)-unit equation, the following property of function field elements turns out to be of importance.

\begin{definition}%
  \label{def:height}
  The height of \(f \in K\) is defined as
  \[H(f) := - \sum_{P \in \mathbb{P}_{K}} \min{\left\{ 0, v_{P}(f) \cdot \deg{P} \right\}}.\]
\end{definition}

Essentially, this definition says that the height of \(f \in K\) is the magnitude of the sum of all negative coefficients in the divisor \((f)\). Theorem~\ref{thm:principal-divisor-degree-zero} says that the negative and positive coefficients in \((f)\) cancel out, so we can also write
\[H(f) = \sum_{P \in \mathbb{P}_{K}} \max{\left\{ 0, v_{P}(f) \cdot \deg{P} \right\}}.\]

The fundamental result that allows the \(S\)-unit equation to be solved is that the height of solutions is bounded. More precisely, we have the following important theorem.

\begin{theorem}%
  \label{thm:height-bound}
  Let \(k\) be an algebraically closed or finite field of characteristic \(p \geq 0\) and \(K / k\) a function field. Consider a finite set \(S \subset \mathbb{P}_{K}\) and \(S\)-units \(\gamma_{1}, \gamma_{2}, \gamma_{3}\) satisfying
  \[\gamma_{1} + \gamma_{2} + \gamma_{3} = 0.\]
  Then we either have that \(\gamma_{1}/\gamma_{3}\) is an element of \(K^{p}\), or
  \[H\left(\frac{\gamma_{1}}{\gamma_{3}}\right) \leq 2g_{K} - 2 + \sum_{P \in S} \deg{P}.\]
  Here \(K^{p}\) is the set of \(p\)th powers in \(K\) for \(p > 0\) and \(K^{0} = k\).
\end{theorem}

This was first shown by Mason~\cite[Lemma 10]{mason-1984-diophantine-equations-over} for function fields \(K / k\) where \(k\) is algebraically closed. Later, Ga\'{a}l and Pohst~\cite{gaal-2006-diophantine-equations-over} expanded this result to function fields over finite fields, so \(k = \mathbb{F}_{q}\) where \(q = p^{n}\) and \(p\) is prime. Silverman~\cite{silverman-1984-the-s-unit-equation} showed that the bound is sharp in the sense that there are function fields where the bound is attained, even for large sets \(S\).

It is worth pointing out that Theorem~\ref{thm:height-bound} also holds for rational functions field \(k(x)\) when \(k\) is not algebraically closed. To see why, note that the irreducible monic polynomials corresponding to the places in a set \(S \subset \mathbb{P}_{k(x)}\) can be factored to obtain irreducible monic polynomials in the rational function field \(\bar{k}(x)\) over the algebraic closure of \(k\). These factors can be combined to form a new set of places \(\bar{S} \subset \mathbb{P}_{\bar{k}(x)}\). The sum of the degrees of the places in \(S\) and \(\bar{S}\) are equal by construction, and since \(g_{k(x)} = g_{\bar{k}(x)} = 0\) we can conclude that
\[2g_{k(x)} - 2 + \sum_{P \in S} \deg{P} = 2g_{\bar{k}(x)} - 2 + \sum_{P \in \bar{S}} \deg{P}.\]
According to Theorem~\ref{thm:height-bound}, the right-hand side of this equality provides a bound on the height of \(\bar{S}\)-units in \(\bar{k}(x)\). By construction of \(\bar{S}\), every \(S\)-unit in \(k(x)\) is also an \(\bar{S}\)-unit in \(\bar{k}(x)\). Therefore, the left-hand side of the equality is a bound on the height of \(S\)-units in \(k(x)\). In other words, Theorem~\ref{thm:height-bound} holds for \(k(x)\).


% -*- TeX-master: "../main.tex" -*-

\chapter{Method}%
\label{chap:method}

In this chapter we discuss our algorithm for solving the \(S\)-unit equation. In Section~\ref{sec:the-algorithm}, we make use of the theory of Chapter~\ref{chap:background} to derive the algorithm, culminating in a precise description in the form of pseudocode. We discuss the asymptotic computational complexity of the algorithm as well as some performance considerations in Section~\ref{sec:complexity-and-performance}

\section{The Algorithm}%
\label{sec:the-algorithm}

The height bound of Theorem~\ref{thm:height-bound} is the fundamental result that allows the \(S\)-unit equation to be solved algorithmically. For an \(S\)-unit \(f\), we have by definition that \(v_{P}(f) = 0\) for all \(P \notin S\). This means that the height of \(f\) is determined by only the finite set of valuations on \(S\). These valuations are discrete, taking on values in \(\mathbb{Z}\). As a result, there are only a finite number of combinations of values for the valuations on \(S\) satisfying a given height bound. Up to a constant factor, each such combination of values corresponds to a unique \(S\)-unit. In this manner, Theorem~\ref{thm:height-bound} limits the space of solutions to the \(S\)-unit equation to a finite set, thus making it approachable by an exhaustive algorithmic search.

In order to determine all \(S\)-units satisfying a given height bound we use the theory of divisors and Riemann-Roch spaces discussed in Section~\ref{sec:divisors-and-riemann-roch-spaces}. Let \(S \subset \mathbb{P}_{K}\) be a finite set of places on the function field \(K\), and let
\[N = 2g_{K} - 2 + \sum_{P \in S} \deg{P}\]
be the bound of Theorem~\ref{thm:height-bound}. Given a positive integer \(n \leq N\), we are interested in finding an \(S\)-unit \(f\) such that \(H(f) = n\). Note that the divisor of \(f\) is a formal sum of places in \(S\). More precisely,
\[(f) = \sum_{P \in S} v_{P}(f) P.\]
If \(S_{0} \subset S\) denotes the zeros of \(f\) and \(S_{\infty} \subset S\) denotes the poles, then we define the \textit{zero divisor} \({(f)}_{0}\) and the \textit{pole divisor} \({(f)}_{\infty}\) by
\[{(f)}_{0} := \sum_{P \in S_{0}} v_{P}(f) P \qquad \text{and} \qquad {(f)}_{\infty} := \sum_{P \in S_{\infty}} - v_{P}(f) P.\]
The requirement that \(H(f) = n\) is then equivalent to saying that \(\deg{{(f)}_{0}} = \deg{{(f)}_{\infty}} = n\).

Reversing this line of reasoning, we are interested in finding integers \(n_{P}\) for \(P \in S\) giving rise to a divisor
\[D = \sum_{P \in S} n_{P} P\]
such that \(\deg{D_{0}} = \deg{D_{\infty}} = n\). Here
\[D_{0} := \sum_{P \in S_{0}} n_{P} P \qquad \text{and} \qquad D_{\infty} := \sum_{P \in S_{\infty}} - n_{P} P,\]
with \(S_{0} \subset S\) denoting the places where \(n_{P} > 0\) and \(S_{\infty} \subset S\) denoting those where \(n_{P} < 0\). Given such \(n_{P}\), we can construct the Riemann-Roch space \(\mathcal{L}(D)\). Following the discussion of Theorem~\ref{thm:riemann-roch-space-properties}, if \(\dim{\mathcal{L}(D)} = 1\) with the single basis vector \(g \in K\), then we can take \(f = g^{-1}\), and by construction \(f\) is an \(S\)-unit satisfying \(H(f) = n\).

Generating the \(S\)-units \(f\) with \(H(f) \leq N\) can thus be reduced to finding all combinations of integers \(n_{P}\) for \(P \in S\) such that
\[\sum_{P \in S_{0}} n_{P} \cdot \deg{P} = \sum_{P \in S_{\infty}} - n_{P} \cdot \deg{P} = n\]
for all positive integers \(n \leq N\). This is similar to the problem of finding \textit{integer partitions} of \(n\). An integer partition of \(n\) into \(m\) parts is a decreasing sequence of positive integers \((x_{1}, x_{2}, \dots, x_{m})\) such that
\[x_{1} + x_{2} + \cdots + x_{m} = n.\]
In our case, we need two integer partitions \((x_{1}, x_{2}, \dots, x_{m})\) and \((y_{1}, y_{2}, \dots, y_{m'})\) of \(n\) with \({m + m' \leq |S|}\). Then, for any order on the places \(S\) such that we can write \(S = \{P_{1}, P_{2}, \dots, P_{|S|}\}\) and \(n_{i} := n_{P_{i}}\), we choose the \(n_{i}\) such that \(n_{i} \cdot \deg{P_{i}} = x_{i}\) for \(i \leq m\), \(n_{i} \cdot \deg{P_{i}} = -y_{i - m}\) for \(m < i \leq m'\), and \(n_{i} = 0\) otherwise. This is possible provided \(\deg{P_{i}} \mid x_{i}\) and \(\deg{P_{j}} \mid y_{j - m}\).

The process of determining the integers \(n_{i}\) is described in Algorithm~\ref{alg:bounded-values}. It considers all appropriately sized pairs of partitions of the bound \(n\). Each pair of partitions gets merged into a sequence \(v\), which is padded with zeros so that it is of length \(|S|\). The algorithm then evaluates every unique permutation of the integers in \(v\), dividing each integer in the permutation by the degree of the corresponding place. The resulting sequence of values is recorded provided each division resulted in an integer.

\begin{algorithm}[htb]
  \caption{An algorithm for generating all possible values for the valuations of an \(S\)-unit of height \(n\). The provided set of places is assumed to be ordered, and we write \(S = \{P_{1}, P_{2}, \dots, P_{|S|}\}\). The result is a set of sequences of values, with the order of the sequences matching those of the places.}%
  \label{alg:bounded-values}
  \begin{algorithmic}[1]
    \Function{Bounded-Values}{$S, n$}
    \State{$V \gets \{\}$} \Comment{Sequences of values found so far.}
    \For{$m = 1, \dots, |S| - 1$}
    \ForAll{$(x_{1}, \dots, x_{m}) \in$ \Call{Partitions}{$n, m$}}
    \For{$m' = 1, \dots, |S| - m$}
    \ForAll{$(y_{1}, \dots, y_{m'}) \in$ \Call{Partitions}{$n, m'$}}
    \State{$v \gets (\underbrace{x_{1}, \dots, x_{m}, -y_{1}, \dots, -y_{m}, 0, \dots, 0}_{|S|\ \text{elements}})$}
    \ForAll{$(n_{1}, \dots, n_{|S|}) \in \Call{Permutations}{v}$}
    \For{$i = 1, \dots, |S|$}
    \State{$n_{i} \gets n_{i} / \deg{P_{i}}$}
    \EndFor{}
    \If{$n_{i} \in \mathbb{Z}\ \text{for all}\ i \in \{1, \dots, |S|\}$}
    \State{$V \gets V \cup (n_{1}, \dots, n_{|S|})$}
    \EndIf{}
    \EndFor{}
    \EndFor{}
    \EndFor{}
    \EndFor{}
    \EndFor{}
    \State{\Return{$V$}}
    \EndFunction{}
  \end{algorithmic}
\end{algorithm}

We can run Algorithm~\ref{alg:bounded-values} for the positive integers \(n \leq N\) to generate every sequence of values corresponding to the valuations of a bounded \(S\)-unit. Each sequence of values leads to a divisor and thus to a Riemann-Roch space. The \(S\)-unit can then be read from the basis of the Riemann-Roch space, provided the space is of dimension 1. This process, which generates every bounded \(S\)-unit, is described in Algorithm~\ref{alg:bounded-s-units}.

\begin{algorithm}[htb]
  \caption{An algorithm for generating all \(S\)-units whose height is bounded by \(N\).}%
  \label{alg:bounded-s-units}
  \begin{algorithmic}[1]
    \Function{Bounded-S-Units}{$S, N$}
    \State{$U \gets \{\}$} \Comment{$S$-units found so far.}
    \For{$n = 1, \dots, N$}
    \ForAll{$(n_{1}, \dots, n_{|S|}) \in$ \Call{Bounded-Values}{$S, n$}}
    \State{$D \gets \sum_{i = 1}^{|S|} n_{i} P_{i}$} \Comment{Construct divisor.}
    \If{$\dim{\mathcal{L}(D) = 1}$}
    \State{$\{f\} \gets \Call{Basis}{\mathcal{L}(D)}$}
    \State{$U \gets U \cup f^{-1}$} \Comment{Found a new $S$-unit.}
    \EndIf{}
    \EndFor{}
    \EndFor{}
    \State{\Return{$U$}}
    \EndFunction{}
  \end{algorithmic}
\end{algorithm}

With the ability to generate all \(S\)-units satisfying the height bound of Theorem~\ref{thm:height-bound}, it is relatively straightforward to find all solutions to the \(S\)-unit equation. Consider any pair of \(S\)-units \(f, g \in K\). We are interested in finding constants \(a, b \in k\) such that
\[af + bg = 1.\]
We can differentiate this equation to obtain
\[af' + bg' = 0.\]
Since \(f\) and \(g\) are \(S\)-units, their derivatives \(f'\) and \(g'\) are nonzero and these two equations form a linear system in the constants \(a\) and \(b\) with at most one solution. Provided a solutions exists, it immediately leads to a solution to the \(S\)-unit equation.

\section{Complexity and Performance}%
\label{sec:complexity-and-performance}

The algorithm presented in Section~\ref{sec:the-algorithm} is effectively a brute force search through the space of bounded \(S\)-units. This is simple to implement, but it is not very efficient since the number of bounded \(S\)-units can quickly grow to be quite large. To see why this is the case, we construct an upper bound on the number of \(S\)-units produced by Algorithm~\ref{alg:bounded-s-units}, depending on the size of the set \(S\) and the height bound \(N\).

First, we note that the number of partitions of the integer \(n\) is given by the \textit{partition function} \(p(n)\). The \textit{restricted partition function} \(p_{m}(n)\) yields the number of partitions of \(n\) into \(m\) parts. There is no known closed-form expression for either function. However, it can be shown that (see e.g.~\cite[Theorem 6.3]{andrews-1984-the-theory-of})
\[p(n) \sim \frac{e^{\pi\sqrt{\frac{2n}{3}}}}{4n\sqrt{3}} \quad \text{as} \quad n \to \infty.\]

Algorithm~\ref{alg:bounded-values} considers all pairs of partitions  of \(n\) into \(m\) and \(m'\) parts with \(m + m' \leq |S|\). The number of such pairs is given by
\[\sum_{m = 1}^{|S| - 1} \left( p_{m}(n) \sum_{m' = 1}^{|S| - m} p_{m'}(n) \right).\]
This expression is somewhat cumbersome. We can simplify our analysis by ignoring the fact that the partitions in Algorithm~\ref{alg:bounded-values} are restricted. There are of course fewer pairs of restricted partitions than pairs of unrestricted partitions. Therefore, \({p(n)}^{2}\) provides a simple upper bound on the number of pairs considered by the algorithm.

Algorithm~\ref{alg:bounded-values} merges each pair of partitions into a sequence of \(|S|\) integers and considers all unique permutations of this sequence. There are at most \(|S|!{}\) unique permutations, so \({p(n)}^{2}|S|!{}\) is an upper bound on the number of sequences produced by Algorithm~\ref{alg:bounded-values}.

In order to generate all \(S\)-units whose height is bounded by \(N\), Algorithm~\ref{alg:bounded-s-units} considers the sequences produced by Algorithm~\ref{alg:bounded-values} for \(n = 1, 2, \dots, N\). Because of the rapid growth of the partition function, the number of sequences is dominated by those for the highest bound \(n = N\). As a result, the number of bounded \(S\)-units generated by Algorithm~\ref{alg:bounded-s-units} is in the order of \({p(N)}^{2}|S|!{}\).

Solving the \(S\)-unit equation requires considering every pair of bounded \(S\)-units. The total number of candidate solutions is therefore in the order of \({({p(N)}^{2}|S|!{})}^{2}\). This grows quickly in both \(|S|\) and \(N\), leaving \(S\)-unit equations for large sets \(S\) and high height bounds \(N\) practically impossible to solve. We note, however, that we do not need to consider every pair. The order of the \(S\)-units in the \(S\)-unit equation does not matter. Additionally, if the pair \((f, g)\) solves the \(S\)-unit equation, then so do \((f^{-1}, - f^{-1}g)\) and \((g^{-1}, - g^{-1}f)\). In practice, we therefore only need to consider one out of every six pairs of \(S\)-units. This does not change the asymptotic behaviour of the algorithm, but it does provide a meaningful improvement in performance.

We can further limit the number of pairs of \(S\)-units the algorithm needs to consider by making use of the fact that in order to solve the \(S\)-unit equation two \(S\)-units must be of the same height. To see why this is the case, note that from property~\ref{prop:valuation-addition} of Theorem~\ref{thm:valuation-properties} it follows that for all \(f, g \in K\) we have
\[H(f + g) \leq H(f) + H(g).\]
Therefore, if \(f + g = 1\), then
\[H(f) = H(1 - g) \leq H(1) + H(g) = H(g)\]
and similarly
\[H(g) = H(1 - f) \leq H(1) + H(f) = H(f).\]
We can conclude that \(H(f) = H(g)\). As with the previous limits on the number of pairs \(S\)-units, this does not alter the time complexity of the algorithm. However, in practice it noticeably speeds up the search for solutions.


% -*- TeX-master: "../main.tex" -*-

\chapter{Results}%
\label{chap:results}

This is an introduction to the chapter.

\section{Examples}%
\label{sec:examples}

To get an impression of the functioning of the Algorithm of Chapter~\ref{chap:method} we consider two examples. First, let \(K = \bar{\mathbb{Q}}(x)\) and
\[S = \left\{ x, x^{-1}, x - 1, x + 2 \right\}.\]
Here we use irreducible polynomials to refer to the corresponding places in \(\mathbb{P}_{K}\) and \(x^{-1}\) denotes the infinite place. From Theorem~\ref{thm:height-bound} we obtain the height bound
\[N = 2g_{K} - 2 + \sum_{P \in S} \deg{P} = 0 - 2 + 4 = 2,\]
so we are interested in \(S\)-units \(f \in \bar{\mathbb{Q}}(x)\) with \(H(f) = 1\) or \(H(f) = 2\). Our implementation of Algorithm~\ref{alg:bounded-s-units} finds \(54\) such \(S\)-units. By considering each pair of the same height and solving the resulting system of equations as discussed in Section~\ref{sec:the-algorithm} we find all solutions to the \(S\)-unit equation. They are given by
\[x - (x - 1) = 1, \quad \frac{x + 2}{2} - \frac{x}{2} = 1, \quad \frac{x + 2}{3} - \frac{x - 1}{3} = 1, \quad \frac{3x}{x + 2} - \frac{2(x - 1)}{x + 2} = 1.\]
Note that each of these solutions is a representative of a group of six solutions as discussed in Section~\ref{sec:complexity-and-performance}.

Let us now consider another example where we take \(K = \mathbb{F}_{5}(x)\) and
\[S = \left\{ x, x^{-1}, x + \bar{2}, x^{2} + x + \bar{1} \right\}.\]
In this case the height bound becomes
\[N = 2g_{K} - 2 + \sum_{P \in S} \deg{P} = 0 - 2 + 5 = 3\]
and Algorithm~\ref{alg:bounded-s-units} finds \(72\) \(S\)-units satisfying this bound. While there are more candidates compared to the previous example, there are only two solutions to the \(S\)-unit equation. They are given by
\[\frac{\bar{3}}{x}  + \frac{x + \bar{2}}{x} = \bar{1}, \quad \frac{\bar{3}(x^{2} + x + \bar{1})}{x^{2}} + \frac{\bar{3}{(x + \bar{2})}^{2}}{x^{2}} = \bar{1}.\]
If we include the place corresponding to \(x - \bar{1}\) in \(S\), the height bound increases by one. The number of bounded \(S\)-units grows much more, however. Algorithm~\ref{alg:bounded-s-units} produces \(658\) candidates and we find \(13\) solutions. This illustrates the outsize effect the number of places in \(S\) has on the number of (candidate) solutions.

\section{An Application in Matroid Theory}%
\label{sec:an-application-in-matroid-theory}

One area where the \(S\)-unit equation plays a role is in matroid theory, specifically in the study of \textit{partial fields}. A partial field \(\mathcal{P}\) is a generalization of the concept of a field and is given by a tuple \((R, G)\) where \(R\) is a ring and \(G \subseteq R^{\times}\) is a multiplicative group with \(-1 \in G\). We say that \(p \in \mathcal{P}\) whenever \(p \in G\) or \(p = 0\). Partial fields are often constructed by taking \(G\) to be the group generated by a finite set of units \(S \subset R^{\times}\), so \(G = \langle S \rangle\). For example, the \textit{dyadic partial field} is given by
\[\mathbb{D} := \left( \mathbb{Z} \left[ \frac{1}{2} \right], \left\langle -1, 2 \right\rangle \right).\]
The elements of \(\mathbb{D}\) are all \(p \in \mathbb{Z} \left[ \frac{1}{2} \right]\) where \(p = \pm 2^{n}\) for \(n \in \mathbb{Z}\). The thesis by van Zwam~\cite{zwam-2009-partial-fields-in} provides more thorough discussion of partial fields and the surrounding theory.

A partial field can be thought of as a field where the addition of two elements is not always defined. Pairs of elements that can be added are therefore of special interest. Notably, an element \(p \in \mathcal{P}\) such that \(p - 1 \in \mathcal{P}\) is called a \textit{fundament element}. Fundamental elements play an important role the study of partial fields and determining all fundamental elements is often necessary. If we consider a partial field \(\mathcal{P} = (K, \langle S \rangle)\) where \(K\) is a function field and \(\langle S \rangle\) denotes the set of \(S\)-units generated by \(S \subset \mathbb{P}_{K}\), then the problem of finding the fundamental elements of \(\mathcal{P}\) is equivalent to finding all pairs \((f, g) \in K^{2}\) solving the \(S\)-unit equation. The algorithm of Chapter~\ref{chap:method} can therefore be applied to find the fundamental elements of certain partial fields.

For an example application of our algorithm in finding fundamental elements we turn to another generalization of fields called \textit{pastures}. Pastures are a more general category than partial fields; every partial field is a pasture but not every pasture is a partial field. We do not discuss pastures in detail here, for that we refer the reader to the work of Baker and Lorscheid~\cite{baker-2020-foundations-of-matroids}.  For our purposes it is sufficient to know that a pasture can be defined in terms of a partial field and a set of equations of the form
\[p + q = 1\]
listing pairs of fundamental elements of the partial field.\todo{Is this true for all or only some pastures?} If all fundamental elements are included in the definition of the pasture, it is itself a partial field.

Appendix A in the work by Baker et al.~\cite{baker-2023-foundations-of-matroids} contains a number of pastures of this form. For example, section A.3.31 discusses a pasture constructed over the partial field
\[\mathbb{F}_{1}^{\pm}(x) = \left( \mathbb{Z} \left[ x, x^{-1} \right], \left\langle -1, x, x^{-1}, x -1, 2x - 1, x^{2} + x - 1 \right\rangle \right)\]
with the equations
\begin{equation*}
  \def\arraystretch{1.5}
  \begin{array}{cc}
    x - (x - 1) = 1,                                                                  & \frac{2x - 1}{x} - \frac{x - 1}{x} = 1,                                    \\
    \frac{x^{2}}{x^{2} + x - 1} + \frac{x - 1}{x^{2} + x - 1} = 1,                    & \frac{2x - 1}{x} - \frac{x - 1}{x} = 1,                                    \\
    \frac{2x - 1}{x^{2} + x - 1} + \frac{x(x - 1)}{x^{2} + x - 1} = 1,                & \frac{x(2x - 1)}{x^{2} + x - 1} - \frac{{(x - 1)}^{2}}{x^{2} + x - 1} = 1, \\
    \frac{x^{3}}{(x - 1)(x^{2} + x - 1)} - \frac{2x - 1}{(x - 1)(x^{2} + x - 1)} = 1. &
  \end{array}
\end{equation*}
We can determine whether these are all fundamental elements of the partial field \(\mathbb{F}_{1}^{\pm}(x)\) by considering the \(S\)-unit equation over the function field \(K = \bar{\mathbb{Q}}(x)\) with
\[S = \left\{ x, x^{-1}, x -1, 2x - 1, x^{2} + x - 1 \right\},\]
where we use irreducible polynomials to refer the the corresponding places in \(\mathbb{P}_{K}\). Any solution involving a constant other than \(\pm 1\) is not valid in the partial field and can therefore be dismissed. With this setup, the algorithm of Chapter~\ref{chap:method} yields all solutions corresponding to the pasture's equations as well as one additional solution given by
\[\frac{x(x^{2} + x - 1)}{{(x - 1)}^{3}} - \frac{{(2x - 1)}^{2}}{{(x - 1)}^{3}} = 1.\]
We can thus conclude that the pasture is not a partial field.


% -*- TeX-master: "../main.tex" -*-

\chapter{Conclusion}%
\label{chap:conclusion}

We have implemented an algorithm for solving the \(S\)-unit equation in function fields over an algebraically closed or finite base field. The algorithm uses the theory of integer partitions to determine all possible divisors of \(S\)-units below the height bound of Theorem~\ref{thm:height-bound}. The \(S\)-units themselves are then constructed from the divisors by considering the basis of their Riemann-Roch space. The existence of such an algorithm was already discussed by Mason in~\cite{mason-1984-diophantine-equations-over}, but no detailed discussion or implementation was publicly available up to now.

We also derived an upper and lower bound on the number of candidate solutions considered by algorithm. Both bounds grow rapidly with the number of places in the set \(S\). As a result, the algorithm is impractical when dealing with \(S\)-units equations over a large set \(S\). However, the algorithm is still useful since problems with relatively few places are already often impossible to solve by hand. An example of such a problem in matroid theory was provided in Section~\ref{sec:an-application-in-matroid-theory}, and Appendix A in~\cite{baker-2024-foundations-of-matroids} contains many more.

Future work might focus on improving the performance of the algorithm by reducing the height bound. In number fields this has been achieved through the use of LLL-reduction and sieve techniques. Similar methods could explored for use in the function field case. Even without such improvements, the algorithm can be applied to solving certain Diophantine equations. It could for example be adopted for use in the method for solving the Thue equation discussed by Mason in~\cite{mason-1984-diophantine-equations-over} and Ga\'{a}l and Pohst in~\cite{gaal-2006-diophantine-equations-over, gaal-2006-diophantine-equations-overa}.


\printbibliography[heading=bibintoc]

\end{document}
