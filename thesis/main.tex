% -*- TeX-master: "main.tex" -*-

\documentclass[a4paper]{report}

\usepackage{amsmath}
\usepackage{amssymb}
\usepackage{amsthm}

\usepackage{enumitem}

% Bibliography.
\usepackage[maxbibnames=8]{biblatex}
\addbibresource{bibliography.bib}

% To-do items.
\usepackage[textwidth=4cm]{todonotes}

\newtheorem{theorem}{Theorem}
\newtheorem{corollary}{Corollary}[theorem]
\newtheorem{lemma}[theorem]{Lemma}

\theoremstyle{definition}
\newtheorem{definition}{Definition}

\theoremstyle{remark}
\newtheorem*{remark}{Remark}

\usepackage{algorithm}
\usepackage{algpseudocode}

% Title page.
\usepackage[english]{babel}
\usepackage{styles/rugscriptie}

\faculty{fse}
\thesistype{Bachelor's Project Mathematics}
\title{Solving the S-Unit Equation in Function Fields}
\author{W.A.~Verschoore de la Houssaije}
\supervisor{O.~Lorscheid}
\supervisor{J.S.~M\"{u}ller}
\date{July 19, 2024}


\begin{document}

\maketitle

\begin{abstract}
  In algebraic number theory, the equation \(x + y = 1\), where \(x\) and \(y\) are units in a certain subring of a field, plays an important role in solving various Diophantine equations as well as in matroid theory. The type of subring containing \(x\) and \(y\) is known as a ring of \(S\)-integers, and the equation is called the \(S\)-unit equation. Much work has been done on developing efficient algorithms for solving the S-unit equation in number fields. Implementations of these algorithms are readily available, for example in the computer algebra package SageMath. While an algorithm for solving the S-unit equation in certain function fields is known, no implementation is publicly available.

  We present an implementation of the algorithm for solving the S-unit equation in function fields. We discuss the theory underlying the algorithm and demonstrate its effectiveness through a number of examples. Notably, our implementation has already proven useful in research in the field of matroid theory.
\end{abstract}

\tableofcontents

% -*- TeX-master: "../main.tex" -*-

\chapter{Introduction}%
\label{chap:introduction}

The equation \(x + y = 1\), where \(x\) and \(y\) are so-called \(S\)-units, plays an important role in algebraic number theory. An \(S\)-unit is a unit the ring of \(S\)-integers, which is a particular kind of subring of a field. Certain Diophantine equations, such as the Thue equation~\cite{thue-1909-uber-annaherungsweerte-algebraischer}, can be reduced to an \(S\)-unit equation. The \(S\)-unit equation also appears in matroid theory, where it corresponds to the problem of finding the fundamental elements of a partial field~\cite{zwam-2009-partial-fields-in}.

The \(S\)-unit equation has been extensively studied in number fields. The theory of linear forms in logarithms can be used to effectively determine an upper bound on the size of solutions to the \(S\)-unit equation~\cite{gyory-1979-on-the-number}. In theory this allows for an exhaustive search for all solutions, but in practice the bounds are often too big for this to be feasible. De Weger introduced a method for reducing the bounds to more reasonable size~\cite{weger-1989-algorithms-for-diophantine}, making use of the LLL-reduction algorithm developed by Lenstra, Lenstra, and Lov\'{a}sz~\cite{lenstra-1982-factoring-polynomials-with}. This method has since been built upon, for example by Alvarado et al.~\cite{alvarado-2021-a-robust-implementation}, whose algorithm is readily available in the computer algebra package SageMath~\cite{sagemath}.

The \(S\)-unit has also been studied in function fields, though to a lesser extent. Mason showed that a similar bound exists on the size of solutions to the \(S\)-unit in function fields over an algebraically closed base field~\cite{mason-1984-diophantine-equations-over}. Ga\'{a}l and Pohst extended this proof to function fields over a finite base field~\cite{gaal-2006-diophantine-equations-over}. This bound naturally leads to an algorithm for solving the \(S\)-unit equation in function fields based on an exhaustive search. However, no detailed description of this algorithm exists, and no implementation is publicly available.

We present a pseudocode description of the algorithm for solving \(S\)-unit equation in function fields as well as an implementation of the algorithm in SageMath. We provide the relevant background theory in Chapter~\ref{chap:background}. The algorithm is discussed in detail in Chapter~\ref{chap:method}. We apply the algorithm to a number of example problems in Chapter~\ref{chap:results}. We provide our conclusions and discuss potential future work in Chapter~\ref{chap:conclusion}.


% -*- TeX-master: "../main.tex" -*-

\chapter{Background}%
\label{chap:background}

Throughout this chapter we denote by \(k\) an arbitrary field. We write \(k(x)\) for the rational function field over \(k\), and we consider a finite algebraic extension \(K\) of \(k(x)\).\todo{Add an overview of upcoming sections.}\todo{Perhaps mention that the first two section closely follow Chapter 1 in~\cite{stichtenoth-2009-algebraic-function-fields} and/or refer the reader to that book for more details.}

\section{Places and Valuations}%
\label{sec:places-and-valuations}

The related notions of places and valuations of function fields are fundamental to the formulation of the \(S\)-unit equation. We first define the concept of a valuation ring.

\begin{definition}%
  \label{def:valuation-ring}
  A valuation ring of the function field \(K\) is a ring \(\mathcal{O} \subset K\) such that for every \(f \in K\) we have that at least one of \(f\) or \(f^{-1}\) is in \(\mathcal{O}\).
\end{definition}

In the rational function field \(k(x)\) we can construct valuation rings by considering an irreducible sonic polynomial \(p \in k[x]\) and taking
\[\mathcal{O}_{p} = \left\{ \frac{f}{g} \,\middle|\, f,g \in k[x] ,\, p \nmid g \right\}.\]
To see that this satisfies Definition~\ref{def:valuation-ring} we note that every element of \(k(x)\) can be written as a reduced fraction \(f/g\) with \(f,g \in k[x]\) and \(p \nmid f\) or \(p \nmid g\). Therefore we always have \(f/g \in \mathcal{O}_{p}\) or \(g/f \in \mathcal{O}_{p}\). We note that a different irreducible monic polynomial \(q \in k[x]\) gives rise to a different valuation ring \(\mathcal{O}_{q}\). For example, we have \(1/q \in \mathcal{O}_{p}\) but \(1/q \notin \mathcal{O}_{q}\).

The units of a valuation ring \(\mathcal{O}\) of the function field \(K\) are those \(f \in K\) such that \(f \in \mathcal{O}\) as well as \(f^{-1} \in \mathcal{O}\). For \(\mathcal{O}_{p}\) in the rational function field \(k(x)\) defined by an irreducible monic polynomial \(p \in k[x]\) this corresponds to the ratios of polynomials which do not contain a factor \(p\) in the numerator nor in the denominator. More precisely,
\[\mathcal{O}_{p}^{\times} = \left\{ \frac{f}{g} \,\middle|\, f,g \in k[x] ,\, p \nmid f ,\, p \nmid g \right\}.\]
This means that the ideal generated by \(p\) is given by \((p) = \mathcal{O}_{p} \setminus \mathcal{O}_{p}^{\times}\). It follows that \((p)\) is the unique maximal ideal of \(\mathcal{O}_{p}\) since any proper ideal \(I \subset \mathcal{O}_{p}\) cannot contain a unit.

This property is not unique to valuation rings of the rational function field generated by an irreducible monic polynomial. The following theorem is a reformulation of Proposition 1.1.5 in~\cite{stichtenoth-2009-algebraic-function-fields}.

\begin{theorem}%
  \label{thm:place-is-maximal-ideal}
  A valuation ring \(\mathcal{O}\) of the function field \(K\) has a unique maximal ideal \(P = \mathcal{O} \setminus \mathcal{O}^{\times}\).
\end{theorem}

The unique maximal ideal \(P\) of a valuation ring \(\mathcal{O}\) is called a \textit{place}. The set of all places in the function field \(K\) is denoted by \(\mathbb{P}_{K}\). Given a place \(P\), we write \(\mathcal{O}_{P}\) for its corresponding valuation ring.

We saw that in the rational function field \(k(x)\) the irreducible monic polynomials correspond to distinct places and thus to distinct valuation rings. These are not the only places in \(k(x)\), however. There is one more valuation ring given by
\[\mathcal{O}_{\infty} = \left\{ \frac{f}{g} \,\middle|\, f,g \in k[x] ,\, \deg{f} \leq \deg{g} \right\},\]
which has the maximal ideal
\[P_{\infty} = \left\{ \frac{f}{g} \,\middle|\, f,g \in k[x] ,\, \deg{f} < \deg{g} \right\}.\]
The place \(P_{\infty}\) is referred to as the \textit{infinite place}. Note that \(P_{\infty} = (1/x)\), so the infinite place is a principal ideal much like the other places.

This, again, is a property common to valuation rings and places of all function fields, not just those of the rational function field. The following theorem is a reformulation of Theorem 1.1.6 in~\cite{stichtenoth-2009-algebraic-function-fields}.

\begin{theorem}%
  \label{thm:place-is-principal-ideal}
  In a function field \(K\) every place \(P \in \mathbb{P}_{K}\) is a principal ideal. Moreover, for a \(p \in \mathcal{O}_{P}\) such that \((p) = P\) we have that any \(f \in K\) can be uniquely written as \(f = p^{n}u\) with \(n \in \mathbb{Z}\) and \(u \in \mathcal{O}_{P}^{\times}\).
\end{theorem}

An element \(p\) which generates the place \(P\) is called a \textit{uniformizer} of \(P\). In the rational function field \(k(x)\) a irreducible monic polynomial is a uniformizer of its corresponding place.

We note that the integer \(n\) in the second part of Theorem~\ref{thm:place-is-principal-ideal} is unique for a given place \(P \in \mathbb{P}_{K}\) and element \(f \in K\). If \(p\) and \(q\) are distinct uniformizers of \(P\) then we have \(p = qs\) and \(q = pt\) for \(s,t \in \mathcal{O}_{P}\). In fact, \(s,t \in \mathcal{O}_{P}^{\times}\) since \(s = p/q\) and \(t = q/p\) are each other's inverses. It follows that if \(f = p^{n}u\) with \(u \in \mathcal{O}_{P}^{\times}\) then also \(f = q^{n}s^{n}u\) with \(s^{n}u \in \mathcal{O}_{P}^{\times}\). This justifies the following definition.

\begin{definition}%
  \label{def:valuation}
  For a place \(P \in \mathbb{P}_{K}\) with a uniformizer \(p\) we construct the function \(v_{P} : K \to \mathbb{Z} \cup \{\infty\}\) by considering for each non-zero \(f \in K\) the unique decomposition \(f = p^{n}u\) as in Theorem~\ref{thm:place-is-principal-ideal} and defining \(v_{P}(f) := n\). Additionally, we define \({v_{P}(0) := \infty}\).
\end{definition}

The function \(v_{P}\) is the \textit{valuation} associated with the place \(P\). In the rational function field \(k(x)\) the valuation \(v_{P}\) for a finite place \(P\) simply counts the number of times the corresponding irreducible monic polynomial divides the numerator of a rational function.

% State theorem saying that this function satisfies the discrete valuation requirements as well as relations P = {...}, O = {...}, Ox = {...}.


% -*- TeX-master: "../main.tex" -*-

\chapter{Method}%
\label{chap:method}

This is an introduction to the chapter.

\section{The Algorithm}%
\label{sec:the-algorithm}

The height bound of Theorem~\ref{thm:height-bound} is the fundamental result that allows the \(S\)-unit equation to be solved algorithmically. For an \(S\)-unit \(f\), we have by definition that \(v_{P}(f) = 0\) for all \(P \notin S\). This means that the height of \(f\) is determined by only the finite set of valuations on \(S\). These valuations are discrete, taking on values in \(\mathbb{Z}\). As a result, there are only a finite number of combinations of values for the valuations on \(S\) satisfying a given height bound. Up to a constant factor, each such combination of values corresponds to a unique \(S\)-unit. In this manner, Theorem~\ref{thm:height-bound} limits the space of solutions to the \(S\)-unit equation to a finite set, thus making it approachable by an exhaustive algorithmic search.

In order to determine all \(S\)-units satisfying a given height bound we use the theory of divisors and Riemann-Roch spaces discussed in Section~\ref{sec:divisors-and-riemann-roch-spaces}. Let \(S \subset \mathbb{P}_{K}\) be a finite set of places on the function field \(K\), and let
\[N = 2g_{K} - 2 + \sum_{P \in S} \deg{P}\]
be the bound of Theorem~\ref{thm:height-bound}. Given a positive integer \(n \leq N\), we are interested in finding an \(S\)-unit \(f\) such that \(H(f) = n\). Note that the divisor of \(f\) is a formal sum of places in \(S\). More precisely,
\[(f) = \sum_{P \in S} v_{P}(f) P.\]
If \(S_{0} \subset S\) denotes the zeros of \(f\) and \(S_{\infty} \subset S\) denotes the poles, then we define the \textit{zero divisor} \({(f)}_{0}\) and the \textit{pole divisor} \({(f)}_{\infty}\) by
\[{(f)}_{0} := \sum_{P \in S_{0}} v_{P}(f) P \qquad \text{and} \qquad {(f)}_{\infty} := \sum_{P \in S_{\infty}} - v_{P}(f) P.\]
The requirement that \(H(f) = n\) is then equivalent to saying that \(\deg{{(f)}_{0}} = \deg{{(f)}_{\infty}} = n\).

Reversing this line of reasoning, we are interested in finding integers \(n_{P}\) for \(P \in S\) giving rise to a divisor
\[D = \sum_{P \in S} n_{P} P\]
such that \(\deg{D_{0}} = \deg{D_{\infty}} = n\). Here
\[D_{0} := \sum_{P \in S_{0}} n_{P} P \qquad \text{and} \qquad D_{\infty} := \sum_{P \in S_{\infty}} - n_{P} P,\]
with \(S_{0} \subset S\) denoting the places where \(n_{P} > 0\) and \(S_{\infty} \subset S\) denoting those where \(n_{P} < 0\). Given such \(n_{P}\), we can construct the Riemann-Roch space \(\mathcal{L}(D)\). Following the discussion of Theorem~\ref{thm:riemann-roch-space-properties}, if \(\dim{\mathcal{L}(D)} = 1\) with the single basis vector \(g \in K\), then we can take \(f = g^{-1}\), and by construction \(f\) is an \(S\)-unit satisfying \(H(f) = n\).

Generating the \(S\)-units \(f\) with \(H(f) \leq N\) can thus be reduced to finding all combinations of integers \(n_{P}\) for \(P \in S\) such that
\[\sum_{P \in S_{0}} n_{P} \cdot \deg{P} = \sum_{P \in S_{\infty}} - n_{P} \cdot \deg{P} = n\]
for all positive integers \(n \leq N\). This is similar to the problem of finding \textit{integer partitions} of \(n\). An integer partition of \(n\) into \(m\) parts is a decreasing sequence of positive integers \((x_{1}, x_{2}, \dots, x_{m})\) such that
\[x_{1} + x_{2} + \cdots + x_{m} = n.\]
In our case, we need two integer partitions \((x_{1}, x_{2}, \dots, x_{m})\) and \((y_{1}, y_{2}, \dots, y_{m'})\) of \(n\) with \({m + m' \leq |S|}\). Then, for any order on the places \(S\) such that we can write \(S = \{P_{1}, P_{2}, \dots, P_{|S|}\}\) and \(n_{i} := n_{P_{i}}\), we choose the \(n_{i}\) such that \(n_{i} \cdot \deg{P_{i}} = x_{i}\) for \(i \leq m\), \(n_{i} \cdot \deg{P_{i}} = -y_{i - m}\) for \(m < i \leq m'\), and \(n_{i} = 0\) otherwise. This is possible provided \(\deg{P_{i}} \mid x_{i}\) and \(\deg{P_{j}} \mid y_{j - m}\).

The process of determining the integers \(n_{i}\) is described in Algorithm~\ref{alg:bounded-values}. It considers all appropriately sized pairs of partitions of the bound \(n\). Each pair of partitions then gets merged into a sequence \(v\), which is padded with zeros so that it is of length \(|S|\). The algorithm then evaluates every unique permutation of the integers in \(v\), dividing each integer in the permutation by the degree of the corresponding place. The resulting sequence of values is recorded provided each division resulted in an integer.

\begin{algorithm}[htb]
  \caption{An algorithm for generating all possible values for the valuations of an \(S\)-unit of height \(n\). The provided set of places is assumed to be ordered, and we write \(S = \{P_{1}, P_{2}, \dots, P_{|S|}\}\). The result is a set of sequences of values, with the order of the sequences matching those of the places.}%
  \label{alg:bounded-values}
  \begin{algorithmic}[1]
    \Function{Bounded-Values}{$S, n$}
    \State{$V \gets \{\}$} \Comment{Sequences of values found so far.}
    \For{$m = 1, \dots, |S| - 1$}
    \ForAll{$(x_{1}, \dots, x_{m}) \in$ \Call{Partitions}{$n, m$}}
    \For{$m' = 1, \dots, |S| - m$}
    \ForAll{$(y_{1}, \dots, y_{m'}) \in$ \Call{Partitions}{$n, m'$}}
    \State{$v \gets (\underbrace{x_{1}, \dots, x_{m}, -y_{1}, \dots, -y_{m}, 0, \dots, 0}_{|S|\ \text{elements}})$}
    \ForAll{$(n_{1}, \dots, n_{|S|}) \in \Call{Permutations}{v}$}
    \For{$i = 1, \dots, |S|$}
    \State{$n_{i} \gets n_{i} / \deg{P_{i}}$}
    \EndFor{}
    \If{$n_{i} \in \mathbb{Z}\ \text{for all}\ i \in \{1, \dots, |S|\}$}
    \State{$V \gets V \cup (n_{1}, \dots, n_{|S|})$}
    \EndIf{}
    \EndFor{}
    \EndFor{}
    \EndFor{}
    \EndFor{}
    \EndFor{}
    \State{\Return{$V$}}
    \EndFunction{}
  \end{algorithmic}
\end{algorithm}

We can run Algorithm~\ref{alg:bounded-values} for the positive integers \(n \leq N\) to generate every sequence of values corresponding to the valuations of a bounded \(S\)-unit. Each sequence of values leads to a divisor and thus to a Riemann-Roch space. The \(S\)-unit can then be read from the basis of the Riemann-Roch space, provided the space is of dimension 1. This process, which generates every bounded \(S\)-unit, is described in Algorithm~\ref{alg:bounded-s-units}.

\begin{algorithm}[htb]
  \caption{An algorithm for generating all \(S\)-units whose height is bounded by \(N\).}%
  \label{alg:bounded-s-units}
  \begin{algorithmic}[1]
    \Function{Bounded-S-Units}{$S, N$}
    \State{$U \gets \{\}$} \Comment{$S$-units found so far.}
    \For{$n = 1, \dots, N$}
    \ForAll{$(n_{1}, \dots, n_{|S|}) \in$ \Call{Bounded-Values}{$S, n$}}
    \State{$D \gets \sum_{i = 1}^{|S|} n_{i} P_{i}$} \Comment{Construct divisor.}
    \If{$\dim{\mathcal{L}(D) = 1}$}
    \State{$\{f\} \gets \Call{Basis}{\mathcal{L}(D)}$}
    \State{$U \gets U \cup f^{-1}$} \Comment{Found a new $S$-unit.}
    \EndIf{}
    \EndFor{}
    \EndFor{}
    \State{\Return{$U$}}
    \EndFunction{}
  \end{algorithmic}
\end{algorithm}

With the ability to generate all \(S\)-units satisfying the height bound of Theorem~\ref{thm:height-bound}, it is relatively straightforward to find all solutions to the \(S\)-unit equation. Consider any pair of \(S\)-units \(f, g \in K\). We are interested in finding constants \(a, b \in k\) such that
\[af + bg = 1.\]
We can differentiate this equation to obtain
\[af' + bg' = 0.\]
Since \(f\) and \(g\) are \(S\)-units, their derivatives \(f'\) and \(g'\) are nonzero and these two equations form a linear system in the constants \(a\) and \(b\) with at most one solution. Provided a solutions exists, it immediately leads to a solution to the \(S\)-unit equation.

\section{Complexity and Performance}%
\label{sec:complexity-and-performance}

The algorithm presented in Section~\ref{sec:the-algorithm} is effectively a brute force search through the space of bounded \(S\)-units. This is simple to implement, but it is not very efficient since the number of bounded \(S\)-units can quickly grow to be quite large. To see why this is the case, we construct an upper bound on the number of \(S\)-units produced by Algorithm~\ref{alg:bounded-s-units}, depending on the size of the set \(S\) and the height bound \(N\).

First, we note that the number of partitions of the integer \(n\) is given by the \textit{partition function} \(p(n)\). The \textit{restricted partition function} \(p_{m}(n)\) yields the number of partitions of \(n\) into \(m\) parts. There is no known closed-form expression for either function. However, it can be shown that\todo{Citation needed.}
\[p(n) \sim \frac{e^{\pi\sqrt{\frac{2n}{3}}}}{4n\sqrt{3}} \quad \text{as} \quad n \to \infty.\]

Algorithm~\ref{alg:bounded-values} considers all pairs of partitions  of \(n\) into \(m\) and \(m'\) parts with \(m + m' \leq |S|\). The number of such pairs is given by
\[\sum_{m = 1}^{|S| - 1} \left( p_{m}(n) \sum_{m' = 1}^{|S| - m} p_{m'}(n) \right).\]
This expression is somewhat cumbersome. We can simplify our analysis by ignoring the fact that the partitions in Algorithm~\ref{alg:bounded-values} are restricted. There are of course fewer pairs of restricted partitions than pairs of unrestricted partitions. Therefore, \({p(n)}^{2}\) provides a simple upper bound on the number of pairs considered by the algorithm.

Algorithm~\ref{alg:bounded-values} merges each pair of partitions into a sequence of \(|S|\) integers and considers all unique permutations of this sequence. There are at most \(|S|!{}\) unique permutations, so \({p(n)}^{2}|S|!{}\) is an upper bound on the number of sequences produced by Algorithm~\ref{alg:bounded-values}.

In order to generate all \(S\)-units whose height is bounded by \(N\), Algorithm~\ref{alg:bounded-s-units} considers the sequences produced by Algorithm~\ref{alg:bounded-values} for \(n = 1, 2, \dots, N\). Because of the rapid growth of the partition function, the number of sequences is dominated by those for the highest bound \(n = N\). As a result, the number of bounded \(S\)-units generated by Algorithm~\ref{alg:bounded-s-units} is in the order of \({p(N)}^{2}|S|!{}\).

Solving the \(S\)-unit equation requires considering every pair of bounded \(S\)-units. The total number of candidate solutions is therefore in the order of \({({p(N)}^{2}|S|!{})}^{2}\). This grows quickly in both \(|S|\) and \(N\), leaving \(S\)-unit equations for large sets \(S\) and high height bounds \(N\) practically impossible to solve. We note, however, that we do not need to consider every pair. The order of the \(S\)-units in the \(S\)-unit equation does not matter. Additionally, if the pair \((f, g)\) solves the \(S\)-unit equation, then so do \((f^{-1}, - f^{-1}g)\) and \((g^{-1}, - g^{-1}f)\). In practice, we therefore only need to consider one out of every six pairs of \(S\)-units. This does not change the asymptotic behaviour of the algorithm, but it does provide a meaningful improvement in performance.

We can further limit the number of pairs of \(S\)-units the algorithm needs to consider by using the fact that in order to solve the \(S\)-unit equation two \(S\)-units must be of the same height. To see why this is the case, note that from property~\ref{prop:valuation-addition} of Theorem~\ref{thm:valuation-properties} it follows that for all \(f, g \in K\) we have
\[H(f + g) \leq H(f) + H(g).\]
Therefore, if \(f + g = 1\), then
\[H(f) = H(1 - g) \leq H(1) + H(g) = H(g)\]
and similarly
\[H(g) = H(1 - f) \leq H(1) + H(f) = H(f).\]
We can conclude that \(H(f) = H(g)\). As with the previous limits on the number of pairs \(S\)-units, this does not alter the time complexity of the algorithm. However, in practice it noticeably speeds up the search for solutions.


% -*- TeX-master: "../main.tex" -*-

\chapter{Results}%
\label{chap:results}

This is an introduction to the chapter.

\section{An Application in Matroid Theory}%
\label{sec:an-application-in-matroid-theory}

One area where the \(S\)-unit equation plays a role is in matroid theory, specifically in the study of \textit{partial fields}. A partial field \(\mathcal{P}\) is a generalization of the concept of a field and is given by a tuple \((R, G)\) where \(R\) is a ring and \(G \subseteq R^{\times}\) is a multiplicative group with \(-1 \in G\). We say that \(p \in \mathcal{P}\) whenever \(p \in G\) or \(p = 0\). Partial fields are often constructed by taking \(G\) to be the group generated by a finite set of units \(S \subset R^{\times}\), so \(G = \langle S \rangle\). For example, the \textit{dyadic partial field} is given by
\[\mathbb{D} := \left( \mathbb{Z} \left[ \frac{1}{2} \right], \left\langle -1, 2 \right\rangle \right).\]
The elements of \(\mathbb{D}\) are all \(p \in \mathbb{Z} \left[ \frac{1}{2} \right]\) where \(p = \pm 2^{n}\) for \(n \in \mathbb{Z}\). The thesis by van Zwam~\cite{zwam-2009-partial-fields-in} provides more thorough discussion of partial fields and the surrounding theory.

A partial field can be thought of as a field where the addition of two elements is not always defined. Pairs of elements that can be added are therefore of special interest. Notably, an element \(p \in \mathcal{P}\) such that \(p - 1 \in \mathcal{P}\) is called a \textit{fundament element}. Fundamental elements play an important role the study of partial fields and determining all fundamental elements is often necessary. If we consider a partial field \(\mathcal{P} = (K, \langle S \rangle)\) where \(K\) is a function field and \(\langle S \rangle\) denotes the set of \(S\)-units generated by \(S \subset \mathbb{P}_{K}\), then the problem of finding the fundamental elements of \(\mathcal{P}\) is equivalent to finding all pairs \((f, g) \in K^{2}\) solving the \(S\)-unit equation. The algorithm of Chapter~\ref{chap:method} can therefore be applied to find the fundamental elements of certain partial fields.

For an example application of our algorithm in finding fundamental elements we turn to another generalization of fields called \textit{pastures}. Pastures are a more general category than partial fields; every partial field is a pasture but not every pasture is a partial field. We do not discuss pastures in detail here, for that we refer the reader to the work of Baker and Lorscheid~\cite{baker-2020-foundations-of-matroids}.  For our purposes it is sufficient to know that a pasture can be defined in terms of a partial field and a set of equations of the form
\[p + q = 1\]
listing pairs of fundamental elements of the partial field.\todo{Is this true for all or only some pastures?} If all fundamental elements are included in the definition of the pasture, it is itself a partial field.

Appendix A in the work by Baker et al.~\cite{baker-2023-foundations-of-matroids} contains a number of pastures of this form. For example, section A.3.31 discusses a pasture constructed over the partial field
\[\mathbb{F}_{1}^{\pm}(x) = \left( \mathbb{Z} \left[ x, x^{-1} \right], \left\langle -1, x, x^{-1}, x -1, 2x - 1, x^{2} + x - 1 \right\rangle \right)\]
with the equations
\begin{equation*}
  \def\arraystretch{1.5}
  \begin{array}{cc}
    x - (x - 1) = 1,                                                                  & \frac{2x - 1}{x} - \frac{x - 1}{x} = 1,                                    \\
    \frac{x^{2}}{x^{2} + x - 1} + \frac{x - 1}{x^{2} + x - 1} = 1,                    & \frac{2x - 1}{x} - \frac{x - 1}{x} = 1,                                    \\
    \frac{2x - 1}{x^{2} + x - 1} + \frac{x(x - 1)}{x^{2} + x - 1} = 1,                & \frac{x(2x - 1)}{x^{2} + x - 1} - \frac{{(x - 1)}^{2}}{x^{2} + x - 1} = 1, \\
    \frac{x^{3}}{(x - 1)(x^{2} + x - 1)} - \frac{2x - 1}{(x - 1)(x^{2} + x - 1)} = 1. &
  \end{array}
\end{equation*}
We can determine whether these are all fundamental elements of the partial field \(\mathbb{F}_{1}^{\pm}(x)\) by considering the \(S\)-unit equation over the function field \(K = \bar{\mathbb{Q}}(x)\) with
\[S = \left\{ x, x^{-1}, x -1, 2x - 1, x^{2} + x - 1 \right\},\]
where we use irreducible polynomials to refer the the corresponding places in \(\mathbb{P}_{K}\). Any solution involving a constant other than \(\pm 1\) is not valid in the partial field and can therefore be dismissed. With this setup, the algorithm of Chapter~\ref{chap:method} yields all solutions corresponding to the pasture's equations as well as one additional solution given by
\[\frac{x(x^{2} + x - 1)}{{(x - 1)}^{3}} - \frac{{(2x - 1)}^{2}}{{(x - 1)}^{3}} = 1.\]
We can thus conclude that the pasture is not a partial field.


% -*- TeX-master: "../main.tex" -*-

\chapter{Conclusion}%
\label{chap:conclusion}

We have implemented an algorithm for solving the \(S\)-unit equation in function fields over an algebraically closed or finite base field. The algorithm uses the theory of integer partitions to determine all possible divisors of \(S\)-units below the height bound of Theorem~\ref{thm:height-bound}. The \(S\)-units themselves are then constructed from the divisors by considering the basis of their Riemann-Roch space. The existence of such an algorithm was already discussed by Mason in~\cite{mason-1984-diophantine-equations-over}, but no detailed discussion or implementation was publicly available up to now.

We also derived an upper bound on the number of candidate solutions considered by algorithm. The upper bound grows rapidly with the height bound as well as the number of places in the set \(S\). As a result, the algorithm is impractical when dealing with \(S\)-units equations over a large set \(S\). However, the algorithm is still useful since problems with relatively few places are already often impossible to solve by hand. An example of such a problem in matroid theory was provided in Section~\ref{sec:an-application-in-matroid-theory}, and Appendix A in~\cite{baker-2023-foundations-of-matroids} contains many more.

Future work might focus on improving the performance of the algorithm by reducing the height bound. In number fields this has been achieved through the use LLL-reduction. Similar techniques could explored for use in the function field case. Even without such improvements the algorithm can be applied to solving certain Diophantine equations. It could for example be adopted for use in the method for solving the Thue equation discussed by Mason in~\cite{mason-1984-diophantine-equations-over} and Ga\'{a}l and Pohst in~\cite{gaal-2006-diophantine-equations-over, gaal-2006-diophantine-equations-overa}.


\printbibliography[heading=bibintoc]

\end{document}
